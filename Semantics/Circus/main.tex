\documentclass{article}
\usepackage{graphicx} % Required for inserting images
\usepackage{circus}

\title{ANN Circus Rules}
\author{Ziggy Attala}
\date{March 2024}

\newcommand{\RC}[1]{\textsf{#1}}
\def \Circus {{\sf\slshape Circus}}

%Declarations, used for the real number circus theory. 
\def \real {\mathbb{R}}
\newcommand{\decimalpoint}{.}

\newcommand{\ubar}[1]{\mkern 1.5mu\underline{\mkern-1.5mu#1\mkern-1.5mu}\mkern 1.5mu}

\newcommand{\meta}[1]{\mathsf{\color{GreyColor}\ubar{#1}}}

\newcommand{\lsem}{\lbrack\!\lbrack}
\newcommand{\rsem}{\rbrack\!\rbrack}

\DeclareFontFamily{U}{mathc}{}
\DeclareFontShape{U}{mathc}{m}{it}%
{<->s*[1.03] mathc10}{}
\DeclareMathAlphabet{\mathscr}{U}{mathc}{m}{it}

\newtheorem{ruleT}{Rule}
\newcounter{tRule}%[section]
\newenvironment{TRule}[2][]{
  \begin{figure*}\centering
    \begin{minipage}{.98\textwidth}
	  \rule{\linewidth}{.1pt}
	  \refstepcounter{tRule}
	  \textbf{Rule~\thetRule.~#1\ {$#2$}}
	  \\
      \noindent\rule{\linewidth}{.1pt}
	  \footnotesize
	  \begin{displaymath}
}
{\end{displaymath}\medskip\noindent\rule{\linewidth}{.1pt}	\end{minipage}\end{figure*}}

\begin{document}

\maketitle

\section{Z Specification of ANNControllers}

\begin{zsection}	 \SECTION spec \parents~standard\_toolkit
\end{zsection}


We just assume we have a context, then we can run the allEvents metalanguage function on this context. 
\begin{zed}
	[ Context ] \\
	ActivationFunction ::= RELU | LINEAR | NOTSPECIFIED \\
	Value == \arithmos \\
	SeqExp ::= null\_ seq | list \ldata \seq \nat \rdata | matrix \ldata \seq \seq Value \rdata | tensor \ldata \seq \seq \seq Value \rdata
\end{zed}

\begin{schema}{ANNParameters}
\\
 layerstructure : SeqExp \\
 weights : SeqExp \\
 biases : SeqExp \\
 activationfunction : ActivationFunction \\
 inputContext : Context \\
 outputContext : Context \\
\end{schema}

\begin{schema}{ANN} 
   annparameters : ANNParameters
\end{schema} 

\begin{schema}{ANNController}
\\
ANN
\end{schema}

\section{Semantic Rules}
NOTES:
\begin{itemize} 
  \item We use $Value$ as a meta-language type, we define it in our Z Specification, but we also use it in our Circus program, they are different, in Circus, it is specialised to the real number type. We keep this, and not just $\real$, to support BNNs, or other types of ANNs, or other precisions of ANNs. 
  \item $\lchanset$ and $\rchanset$ brackets, used in rule 16, for example, in the metalanguage, for us, it is in the target language, but how do we express, the set in between, it is all the elements, in the set in between, technically, one at a time. Syntax inspired by connevts from rule 14, in target language is brackets, and inside is the set, that returns a set of events. 
  \item Previous rules used: Rule 4, defined on page 36, of the robochart reference manual, $allEvents(c : Context) : Set(Event)$. 
  \item in TRule environment, latex, cannot make new lines, in second argument of environment, goes off edge in Rule 6, chansplit, metafunction declaration. Will fix, just noting. 
  \item Just using c.name for name of process, what should it be, fully qualified name, to link to other semantic components, or will the process be renamed in circus? 
  \item The only semantic rule we will have to implement, that is not here, in epsilon, is $allEvents$, Rule 4.
  \item As the RC semantics, we assume the existence of $eventId(e: Event)$, unique identifiers for the name of the event. 
  \item for all, needs to be in order, it is a set of events, needs to be ordered. the implementation is ordered, we need to have an order, defined by the user, replicated according to an ordering function. That can be implemented, assume existence of $order$ function, implemented by the implemetnation as a list. 
  \item going to assume the existence of an $order(s : \power Event): \seq Event$ function, that returns a sequence of events, of the same size of the set of events, can be implemented by lists, in Eclipse, by ordered, containement lists. 
\end{itemize} 

\section{Semantics (with separate channels)} 

This is the semantics with a separate channel for each communication, as fits with Circus without partial channel instnatiation. 

Also, in this semantics, we extract the values of the weights and biases individually, not using a function. 



We specify our translation rules as functions from RoboChart types to \Circus \ types. We use the \Circus \ types as presented in Appendix A of Oliveira's thesis \cite{Oli06} in BNF form. We require a minimal subset of types of RoboChart to specify our translation rules, and we specify these in Z, displayed in Figure \ref{fig:robochart-ann-types}. These are to establish a clear basis for the translation rules in our meta-language and to define some helpful shorthand to simplify the description of these rules. 

\begin{figure}
    \begin{zed}
	[ Context ] \\
	ActivationFunction ::= RELU | LINEAR | NOTSPECIFIED \\
	Value == \arithmos \\
	SeqExp ::= null\_ seq | list \ldata \seq \nat \rdata | matrix \ldata \seq \seq Value \rdata | tensor \ldata \seq \seq \seq Value \rdata
\end{zed}

\begin{schema}{ANNParameters}
\\
 layerstructure : SeqExp \\
 weights : SeqExp \\
 biases : SeqExp \\
 activationfunction : ActivationFunction \\
 inputContext : Context \\
 outputContext : Context \\
\end{schema}

\begin{schema}{ANN} 
   annparameters : ANNParameters
\end{schema} 

\begin{schema}{ANNController}
\\
ANN
\end{schema}
    \caption{RoboChart types for ANN translations, specified in Z.} 
    \label{fig:robochart-ann-types}
\end{figure}

First, declare the type $Context$ to represent the RoboChart \RC{Context} class; we do not require any information from this type apart from its declaration for our translation rules. We also declare this type to link to the $allEvents$ functions defined in the RoboChart reference manual \cite{RoboChart}. Next, we define \RC{ActivationFunction} using a simple free type in Z with three constants. We define $Value$ as a type synonym for the type of values communicated by our abstract ANNs, known as the arithmos type. We consider the real numbers functionally, but we define a general form of ANN, which can operate on any arithmetic type. We define $SeqExp$, a type in RoboChart representing a sequence expression, using three constructors: $list$, $matrix$, and $tensor$. For convenience, we define $list$ as a sequence of natural numbers, as we only require natural number sequences for our ANN components. We use $matrix$ and $tensor$ as shorthand for double or triple nested sequences, as we capture matrices and (3D) tensors using nested sequences. Defining $SeqExp$ this way enables a more convenient representation of our translation rules. 

We capture the types of our ANN classes using simple Z schemas, which we use to enable syntax matching with our EMF models defining the RoboChart metamodel. First, $ANNParameters$ is a schema with a declared variable for each element of \RC{ANNParameters}. Finally, we define $ANN$ and $ANNController$ as schemas to support syntax close to the RoboChart metamodel. 


%We can now define our syntax translation rules using a meta-language using these types. In this language, elements of the meta-language are underlined, and elements of the target language, \Circus \, are in italics. The elements of the meta-language are evaluated by further calls to rules or resolving variables of the meta-language into the target language, while elements of the target can be seen as constants elements of the functions. We have implemented these functions using the Epsilon EOL language for model-to-model transformations \footnote{\url{eclipse.dev/epsilon/doc/eol/}}.

\begin{TRule}[Semantics of ANNs \\]{ \meta{\lsem c : ANNController \rsem_{
    \mathscr{ANN}} : Program =}}
  \begin{array}[t]{l}
    \meta{ANNChannelDecl(c)} \\%
    \meta{ANNProc(c)} \\%
  \end{array} \\%
  \label{rule:ANNsemantics}
\end{TRule} 

We begin with Rule \ref{rule:ANNsemantics}, our top-level rule. This rule uses the semantic brackets, takes any $ANNController$, and returns semantics that may be any number of \Circus \ paragraphs, such as channel declarations, channel set declarations, and process declarations. We represent this using the $Program$ type, the top-level element of the \Circus \ AST. This rule calls two further meta-language rules: $ANNChannelDecl$ and $ANNProc$. 


\begin{TRule}[Function ANNChannelDecl \\]{\meta{ANNChannelDecl(c : ANNController) : Program =}}
  \begin{array}[t]{l}
    \meta{ANNChannels(c, layerNo(c), lastLayerS)} \\
    \circchannel terminate \\%
    \circchannelset ANNHiddenEvts == \lchanset \meta{hiddenEvts} \rchanset
  \end{array} \\%
  
  \meta{\bf where} \\%
  \begin{array}[t]{l}
    \meta{hiddenEvts = \{ l, n : \nat | (0 < l < layerNo) \land n \in 1 \upto layerSize(l) \spot layerRes(l,n) \} } \\%
     \meta{lastLayerS = last~ ((list \inv) c.annparameters.layerstructure)} \\%
  \end{array} 
  \label{rule:annchanneldecl}
\end{TRule} 

Rule \ref{rule:annchanneldecl} is the top-level rule for defining the channels required by our semantics. We first call the meta-language rule $ANNChannels$, which represents the variable channels: the channels that are specific to each ANN component. We declare one channel, $terminate$, that can be seen as a constant channel in our meta-language function and represents termination. An ANN component can never choose to engage in this event itself, but it needs to be able to engage in it to synchronise with other RoboChart components. We also declare a channel set $ANNHiddenEvts$, representing all the channels that should be hidden in the complete behaviour of our semantics. It is defined by a variable in our meta-language: $hiddenEvts$. 

We define the variable $hiddenEvts$ using a set comprehension expression. This variable defines all channels associated with the hidden layers that should be hidden in the overall behaviour of the process. We define this variable using a set comprehension expression, using local variables $l$ representing the layer and $n$ representing the node index. We limit the values $l$ and $n$, which can be taken using the predicate part of this set comprehension expression, and the expression we form defines our channel using the $layerRes$ meta-language rule, which returns a channel declaration. Using this, we obtain all the channels that should be hidden in our semantics. 

We define the variable $lastLayerS$, the size of the last layer, using the inverse of the $list$ constructor of our free type for sequence expressions in Figure \ref{fig:robochart-ann-types}. We use this to obtain a sequence of natural numbers, and we then get the last element using the $last$ function, itself defined in the Z mathematical toolkit. 

\begin{TRule}[Function ANNChannels \\]{\meta{ANNChannels(c : ANNController, l, n : \nat) : Program =}}
  \begin{array}[t]{l}
    \meta{\bf{if (l == 0) \land (n == 1)}} \\
      
    \meta{\bf{then}} \\
    \t1 %
    \begin{array}[t]{l}
      \circchannel \meta{layerRes(l, n)}
    \end{array} \\%
    \meta{\bf{else}}  \\%
    \t1 %
    \begin{array}[t]{l}
      \circchannel \meta{layerRes(1, n)} \\%
      \meta{if(l \neq 0)} \\%
      \t1 % 
      \meta{NodeOutChannels(l, n, LStructure(c, (l - 1)))} \\%
      \meta{if (n > 1)} \\%
      \t1 %
      \begin{array}[t]{l}
        \meta{ANNChannels(l, (n-1))}
      \end{array}  \\
      
      \meta{else} \\
      \t1 %
      \begin{array}[t]{l}
        \meta{ANNChannels((l-1), LStructure(c, (l-1))}
      \end{array} \\
    \end{array} \\
    
  \end{array} \\%
  \label{rule:annchannels}
\end{TRule} 

Rule \ref{rule:annchannels} defines the function that represents all the variable channels in our semantics as a recursive meta-language function. This function has three parameters: $c$, an $ANNController$, $l$, representing the layer of the ANN component we are considering, and $n$, the node index of the ANN component. This function computes all channels by proceeding backwards from the output layer and the last node back to the first node in the input layer. 

The base case is when $l$ is $0$, and $n$ is $1$ because we use $0$ to represent the input layer, which has defined channels but no behaviour. We capture the input layer in a similar way to an interface. We use one-based indexed for the nodes, as we do not have the concept of an input node. In the base case, we return just the channel $layerRes.0.1$, as this is the channel associated with the first node of the input layer. 

In the recursive case, we first define the channel representing the layer output of the current layer and node, $layerRes(l,n)$. Then, if we do not consider the channels of an input layer, that is, if $l \neq 0$, then we need to consider the channels of the node itself, which we define through the meta-language function $NodeOutChannels$, which operates in a similar way to this function. The recursive call changes if we call the previous node in this layer, $n > 1$, or if we call the previous layer when $n = 0$. 


\begin{TRule}[Function NodeOutChannels \\]{\meta{NodeOutChannels(l, n, i : \nat) : Program =}}
  \begin{array}[t]{l}
    \meta{\bf{if (i == 1)}} \\%
    \meta{\bf{then}} \\
    \t1 %
    \begin{array}[t]{l}
      \circchannel \meta{nodeOut(l, n, i)}
    \end{array} \\%
    \meta{\bf{else}}  \\%
    \t1 %
    \begin{array}[t]{l}
      \circchannel \meta{nodeOut(1, n, i)} \\%
      \meta{NodeOutChannels(l, n, (i-1))} \\%
    \end{array} \\%
    
  \end{array} \\%
  \label{rule:nodeoutchannels}
\end{TRule} 

Rule \ref{rule:nodeoutchannels} functions in a very similar way to Rule \ref{rule:annchannels}. We need these channels to enable intra-node communication between the processes representing the input to this node and the collator of this node. Here, $i$ represents the input index, the size of the previous layer's output, used to transmit the results from the last layer to the next layer. 


\begin{TRule}[Function ANNProc \\]{\meta{ANNProc(c : ANNController) : ProcDecl =}}
  \begin{array}[t]{l} 
    \circprocess \meta{c.name} \circdef \\%
    \t1 %
    \begin{array}[t]{l}
      \circbegin \\%
      Collator \circdef l, n, i : \nat; sum : Value \circspot %
      \begin{array}[t]{l}
	    \meta{Collator(c)} \\%
      \end{array} \\%
      NodeIn \circdef l, n, i : \nat \circspot %
      \begin{array}[t]{l}
	    \meta{NodeIn(c)} \\%
      \end{array} \\%
      Node \circdef l, n, inpSize : \nat \circspot %
      \begin{array}[t]{l}
	    \meta{Node(c)} \\%
      \end{array} \\%
      HiddenLayer \circdef l, s, inpSize : \nat \circspot %
      \begin{array}[t]{l} 
        \meta{HiddenLayer(c)}
      \end{array} \\%
      HiddenLayers \circdef \meta{HiddenLayers(c, layerNo(c) - 1)} \\%
      OutputLayer \circdef \meta{OutputLayer(c)} \\% 
      ANN \circdef %
      \begin{array}[t]{l}
        ((HiddenLayers \lpar | \meta{IndexedLayerRes(layerNo(c)-1)} | \rpar OutputLayer) \\%
         \circhide ANNHiddenEvts) \circseq ANN \\% 
      \end{array} \\%
      
      ANNRenamed \circdef \meta{ANNRenamed(c)} \\%
      
      \circspot ANNRenamed \\%
      
      \circend \\
    \end{array}
    
  \end{array} \\%
        
  \label{rule:annproc}
\end{TRule} 

We now consider the rules defining our ANN model's process paragraph. We start with the top-level process rule $ANNProc$ in Rule \ref{rule:annproc}. In a paragraph, we define multiple actions, similar to CSP processes, then a main action after the syntax $\circspot$, which represents the behaviour of this process. We name our process using the $name$ attribute of our $ANNController$. We have access to this attribute as an \RC{ANNController} is a \RC{NamedElement} in RoboChart. In our first four actions, $Collator$, $NodeIn$, $Node$, and $HiddenLayer$, we have fixed local variables represented by non-underlined syntax in our meta-language, followed by a meta-language function representing the ANN-specific definition of this action. 

For the last four definitions, we do not use local variable definitions as the behaviour of these actions represents the overall structure of an ANN pattern; their behaviour is less influenced by the hyperparameters of an ANN model. $HiddenLayers$ and $OutputLayer$ are each defined by a meta-language function capturing the composition of the hidden layers and the behaviour of just the output layer in isolation. The action $ANN$ defines an ANN component's main behaviour: the repeating parallel composition of $HiddenLayers$ and $OutputLayer$, with $ANNHiddenEvts$ hidden. We capture the synchronisation set of this parallel composition using the $IndexedLayerRes(layerNo(c) - 1)$ meta-function, which returns all channels in the penultimate layer, where $layerNo(c)$ is the number of layers in the ANN model. We use the function $ANNRenamed$ to capture the contextual renaming of our ANN component to events of the RoboChart model. Finally, our main action is defined as $ANNRenamed$, which captures the behaviour of our ANN semantics. 

\begin{TRule}[Function Collator \\]{\meta{Collator(c : ANNController) :\\%
 CSPAction =}}
  \begin{array}[t]{l} 
    \meta{ \Extchoice l : 1 \upto layerNo(c); n : 1 \upto LStructure(c, l); i : 0 \upto LStructure(c, (l-1))} \\%
    \t1 %
    \begin{array}[t]{l}
      
      \meta{ @} %
      \lcircguard l = \meta{l} \land n = \meta{n} \land i = \meta{i} \rcircguard \circguard \\%
      \t1 %
      \begin{array}[t]{l}
        \meta{\bf{if (i == 0)}}  \\
        \meta{\bf{then}} \\
        \t1 %
          \meta{layerRes(l,n)}!relu(sum + (\meta{bias(c, l, n)})) \then \Skip \\%
    \meta{\bf{else}} \\%
    \t1 %
        \meta{nodeOut(l, n, i)}?x \then Collator(l, n, (i-1), (sum + x))
      \end{array}
    \end{array}
  \end{array} \\%
  \label{rule:collator}
\end{TRule} 

Rule \ref{rule:collator} defines the rule for the behaviour of the $Collator$ action, represented using the $CSPAction$ AST element. This rule defines, at its core, a recursive action that sums all values communicated by the channel $nodeOut$ and then communicates these results through the $layerRes(l,n)$ channel as its base case, with the $bias$ from the $bias$ meta-function. In the target language, this recursive action is also called \textit{Collator} and is shown here as a constant element in the recursive case in this rule. The constructs surrounding this behaviour are a distributed external choice over $l$, the layer size, $n$, the node size, and $i$, the input size, the size of the previous layer. Each is guarded by checking that the target language variable is equal to the meta-language variable, meaning only one behaviour is possible at every parameter instantiation. This structure is only needed to support the creation of distinct channels for every layer, node, and input size, as is required in \Circus. 



\begin{TRule}[Function NodeIn \\]{\meta{NodeIn(c : ANNController) :\\%
 CSPAction =}}
  \begin{array}[t]{l} 
    \meta{ \Extchoice l : 1 \upto layerNo(c); n : 1 \upto LStructure(c, l); i : 1 \upto LStructure(c, (l-1))} \\%
    \t1 %
    \begin{array}[t]{l}
      
      \meta{ @} %
      \lcircguard l = \meta{l} \land n = \meta{n} \land i = \meta{i} \rcircguard \circguard \\%
      \t1 %
      \meta{layerRes(l-1,i)}?x \then \meta{nodeOut(l, n, i)}!(x * \meta{weight(c, l, n, i)}) \then \Skip 
    \end{array}
  \end{array} \\%
  \label{rule:nodein}
\end{TRule} 

Rule \ref{rule:nodein} defines the behaviour of the $NodeIn$ action: to receive the value communicated through the channel $layerRes(l-1, i)$, then to transmit a modified version of this using the channel $nodeOut(l,n,i)$. It is a buffer that applies the appropriate weight to the value communicated by the previous layer's event. It is a modifying buffer. In our \Circus \ model of an ANN's behaviour, each $NodeIn$ action is used to transmit the result from the previous layer onto the current layer. We use the $nodeOut$ channel as an intermediate communication channel, on which every $Collator$ process synchronises to obtain the weighted output of the previous layer. We use a surrounding structure identical to that of Rule \ref{rule:collator} to support distinct channel naming. 
 
\begin{TRule}[Function Node \\]{\meta{Node(c : ANNController) : CSPAction =}}
  \begin{array}[t]{l} 
    \meta{ \Extchoice l : 1 \upto layerNo(c); n : 1 \upto LStructure(c, l) } \\%
    \t1 %
    \begin{array}[t]{l}
      
      \meta{ @} %
      \lcircguard l = \meta{l} \land n = \meta{n} \rcircguard \circguard %
      
	  \begin{array}[t]{l} 
        ((\Interleave i: 1 \upto inpSize \circspot NodeIn(l, n, i)) \\%
        \lpar | \lchanset \meta{IndexedNodeOut(l, n)} \rchanset | \rpar \\%
        Collator(l, n, inpSize, 0) \circhide \lchanset \meta{IndexedNodeOut(l, n)} \rchanset )
      \end{array}
      
    \end{array}
  \end{array} \\%
  \label{rule:node}
\end{TRule} 

Similar to Rule \ref{rule:nodein} and Rule \ref{rule:collator}, Rule \ref{rule:node} uses the external choice and guarding considering the $l$ and $n$ indices to support the distinct channel naming. The parallel composition of two actions defines this rule. First, a distributed interleaving of all $NodeIn$ processes indexed by the layer, node, and considering every input from $inpSize$. Here, $inpSize$ is one of the parameters for $Node$ in the target language; see Rule \ref{rule:annproc}. This entire action is defined in the target language, as its definition does not change based on the parameters of $Node$. The second action is the $Collator$ action to receive all communications from the $NodeIn$ interleavings and transmit the result on the $layerRes$ channel. We use $IndexedNodeOut(l,n)$ to define the channels we synchronise on. This meta-function returns all $nodeOut$ channels associated with this node: $nodeOut.l.n.i$ where $i$ is the size of the previous layer. We use the channel set operator, denoted using $\lchanset \rchanset$, of these channels to define all events that can be communicated using this set of channels. Finally, we hide this synchronisation set from the resultant process, so the internal communication channel $nodeOut$ is not visible in the $Node$ action. 

\begin{TRule}[Function HiddenLayer \\]{\meta{HiddenLayer(c : ANNController) : CSPAction =}}
  \begin{array}[t]{l} 
    \meta{ \Extchoice l : 1 \upto layerNo(c)} \\%
    \t1 %
    \begin{array}[t]{l}
      
      \meta{ @} %
      \lcircguard l = \meta{l} \rcircguard \circguard %
        (\lpar \lchanset \meta{IndexedLayerRes(l-1)} \rchanset \rpar i : 1 \upto s \circspot Node(l, i, inpSize))
    \end{array}
  \end{array} \\%
  \label{rule:hiddenlayer}
\end{TRule} 

Rule \ref{rule:hiddenlayer} defines the behaviour of our $HiddenLayer$ action. We define this using a replicated parallel, synchronised on the channel set of all $layerRes$ channels associated with this layer, captured by the meta-function $IndexedLayerRes(l-1)$. A layer in an ANN model synchronises with the previous layer's result ($l-1$) because these events are shared across every node in this layer. Each node process engages with a single event indexed by $layerRes.l$ to communicate its output; these communications do not need to be synchronised as each node has a unique output channel. Using this rule, Rule \ref{rule:hiddenlayers} defines the behaviour of an arbitrary number of hidden numbers as a recursive composition of parallel actions. 

\begin{TRule}[Function HiddenLayers \\]{\meta{HiddenLayers(c : ANNController, l : \nat) : CSPAction =}}
  \begin{array}[t]{l} 
      \meta{\bf{if(l == 1)}} \\%
      \meta{\bf{then}} \\%
      \t1 %
      \begin{array}[t]{l}
        (HiddenLayer(\meta{l}, \meta{LStructure(l)}, \meta{LStructure(l-1)}))
      \end{array} \\%
      \meta{\bf{else}} \\%
      \t1 %
      \begin{array}[t]{l}
        (\meta{HiddenLayers(c, (l-1))} \\%
        \lpar | \lchanset \meta{IndexedLayerRes(c, l-1)} \rchanset | \rpar \\%
        HiddenLayer(\meta{l}, \meta{LStructure(l)}, \meta{LStructure(l-1)})
      \end{array}
  \end{array} \\%
  \label{rule:hiddenlayers}
\end{TRule} 

\begin{TRule}[Function OutputLayer \\]{\meta{OutputLayer(c : ANNController) : CSPAction =}}
  \begin{array}[t]{l} 
    \lpar \lchanset \meta{IndexedLayerRes(layerNo(c)-1)} \rchanset \rpar i : 1 \upto \meta{LStructure(layerNo(c))} \circspot \\%
   \t1 Node(l, i, \meta{LStructure(layerNo(c)-1)}))
  \end{array} \\%
  \label{rule:outputlayer}
\end{TRule} 

Rule \ref{rule:outputlayer} defines the output layer for our ANN model. This layer can be defined as a layer whose parameters are constant, using the meta-function $layerNo(c)$ as a constant to define the shape of our output layer. We separate the hidden and output layers of an ANN model to capture the standard pattern used when designing an ANN model. There is a fixed shape to the hidden layers, for example, $6$ layers with $50$ nodes in each \cite{CAV2019}, and the shape of both the input and output layers are distinct from this shape, for example, $5$ and $5$. Moreover, the shape of the input and output layers are defined by the context where an ANN model is used, but the shape of the hidden layers is defined by the complexity of the desired relation, which is independent of context and is a training consideration. Finally, output layers often have probabilistic interpretations, and their results are subject to additional external analysis, which we can support easier with a separate defined output layer action in \Circus. 

\begin{TRule}[Function ANNRenamed \\]{\meta{ANNRenamed(c : ANNController) : CSPAction) = }} 
  \begin{array}[t]{l}
    (ANN) %
        \begin{array}[t]{l}
          \lcircrename \meta{orderedLayerRes} := \meta{eventList} \rcircrename %
          \circinterrupt ~ terminate \then \Skip \\%
        \end{array} \\%
  \end{array} \\%
  \meta{\bf{where}}  \\%
  \begin{array}[t]{l}
    \meta{orderedLayerRes =} \\%
    \t1 \meta{order(\{ l : \{0, layerNo(c)\} ; n : 1 \upto LStructure(c, l)} \meta{ @ layerRes(l,n)\})} \\%
    \meta{eventList = order(allEvents(c.annparameters.inputContext)) \cat }\\%
    \t1 \meta{order(allEvents(c.annparameters.outputContext)) }
  \end{array}
  \label{rule:annrenamed}
\end{TRule}

We present the last of our functional rules that define the process presented in Rule \ref{rule:annproc}, in Rule \ref{rule:annrenamed}. This rule defines the target language action $ANN$, see Rule \ref{rule:annproc}, where a renaming is applied, is interrupted ($\circinterrupt$) by the action $terminate \then \Skip$. This is to support if the other components in the system terminate, represented by the event $terminate$, then the ANN component should terminate and behave as $\Skip$. The renaming operation is to rename the $layerRes$ channels to the contextual channels defined by the user in the $inputContext$ and $outputContext$ in RoboChart. We define that all events in $orderedLayerRes$ are renamed to the events from $eventList$, the contextual events. Here, we define $orderedLayerRes$ by a set comprehension expression $layerRes(l,n)$, where $l$ is either $0$ or $layerNo$ (input or output layer), and $n$ is all nodes in these layers. We define $eventList$ as all events, using the $allEvents$ meta-function \cite{RoboChart}, of the input context concatenated with the events of the output context. The $order$ function is then applied to both of these sets, here we assume this function takes a set and creates a sequence ordering this set, we can implement this function in EMF as the references of an object is defined using lists. In other words, our implementation is based on the order in which the events appear in the textual language for RoboChart, so in our current implementation, the ordering of events is an important consideration for ANN components. 

\begin{TRule}[Function LStructure \\]{\meta{LStructure(c : ANNController, i : \nat) : \nat =}}
  \begin{array}[t]{l}
    \meta{\bf{if(i == 0)}} \\
    \meta{\bf{then}} \\
    \t1 %
	  \meta{ \# allEvents(c.annparameters.inputContext) } \\
    \meta{\bf{else}} \\
    \t1 %
      \meta{((list \inv) c.annparameters.layerstructure)~i}
  \end{array} 
  \label{rule:lstructure}
\end{TRule} 

The explanation of the rules that define the functional behaviour of the semantics for our ANN components is now complete. Rules \ref{rule:layerres} through to our final rule \ref{rule:indexednodeout} present helper functions for values, channels, and channel sets used to define the behaviour of our rules. These rules also enable a cleaner syntax for presenting the functional rules. 

Rule \ref{rule:lstructure} and Rule \ref{rule:layerno} define constants about our ANN model. We define $LStructure$ as the size of each layer, using the inverse of the $list$ free type constructor indexed by $i$ and the number of events in the input context if we refer to the input layer ($l = 0$). The function $layerNo$ defines the number of layers in our model, simply the size of the $layerstructure$ sequence in our $ANNController$. 

Next, we discuss our helper rules that refer to channels, a $CSExp$ in \Circus. Our basic channels are defined in Rule \ref{rule:layerres} and Rule \ref{rule:nodeout}, defining $layerRes$ with $l$ and $n$, and $nodeOut$ indexed with $l$, $n$, and $i$. We define the indexed versions of these channels in Rules \ref{rule:indexedlayerres} and \ref{rule:indexednodeout}. The indexed versions take all but the last parameter and return the channel set containing all channels constructed using the $l$ parameter for $layerRes$ and the $l$ and $n$ parameters for $nodeOut$. Finally, for the $nodeOut$ channel only, we define $AllNodeOut$ in Rule \ref{rule:allnodeout}. This function is a shorthand for the channel set of $nodeOut$ channels used by the $ANNController$ $c$. To describe this, we define the number of layers, for $l$, by $layerNo$, the size of each layer, for $n$, by $LStructure(l)$, and the size of the previous layer, for the index $i$, set by $LStructure(l-1)$. 



\begin{TRule}[layerRes Channel \\]{\meta{layerRes(l : \nat, n : \nat) : CSExp =}}
  \begin{array}[t]{l}
    layerRes\meta{ln} : Value
  \end{array} 
  \label{rule:layerres}
\end{TRule} 

\begin{TRule}[Function IndexedLayerRes \\]{\meta{IndexedLayerRes(l : \nat) : CSExp =}}
  \begin{array}[t]{l}
    \lchanset \meta{\{ n : 1 \upto {LStructure(l)} @ layerRes(l, n)\}}
    \rchanset 
  \end{array} 
  \label{rule:indexedlayerres}
\end{TRule} 

\begin{TRule}[nodeOut Channel \\]{\meta{nodeOut(l : \nat, n : \nat, i : \nat) : CSExp =}}
  \begin{array}[t]{l}
    nodeOut\meta{lni} : Value
  \end{array} 
  \label{rule:nodeout}
\end{TRule} 

\begin{TRule}[Function layerNo (number of layers) \\]{\meta{layerNo(c : ANNController) : \nat =}}
  \begin{array}[t]{l}
    \meta{\# ((list \inv)~c.annparameters.layerstructure)}
  \end{array} 
  \label{rule:layerno}
\end{TRule} 

\begin{TRule}[Function weight \\]{\meta{weight(c : ANNController; l, n, i : \nat) : Value =}}
  \begin{array}[t]{l}
    \meta{((tensor \inv)~c.annparameters.weights)~l~n~i}
  \end{array} 
  \label{rule:weight}
\end{TRule} 

\begin{TRule}[Function bias \\]{\meta{bias(c : ANNController; l, n : \nat) : Value =}}
  \begin{array}[t]{l}
    \meta{((matrix \inv)~c.annparameters.biases)~l~n}
  \end{array} 
  \label{rule:bias}
\end{TRule} 

\begin{TRule}[Function AllNodeOut \\]{\meta{AllNodeOut(c : ANNController) : CSExp =}}
  \begin{array}[t]{l}
    \lchanset \meta{\{l : 1 \upto layerNo(c) ; n : 1 \upto LStructure(c, l) ; i : 1 \upto LStructure(c, (l-1)) @} \\
    \t1 %
    \meta{nodeOut(l,n,i) \}} \rchanset 
  \end{array} \\%
  \label{rule:allnodeout}
\end{TRule} 

\begin{TRule}[Function IndexedNodeOut \\]{\meta{IndexedNodeOut(c : ANNController, l, n : \nat) : CSExp =}}
  \begin{array}[t]{l}
    \lchanset \meta{ \{i : 1 \upto LStructure(c, l-1) @ nodeOut(l,n,i) \} } \rchanset
  \end{array}
  \label{rule:indexednodeout}
\end{TRule} 

Finally, we define our trained parameters, the weights and biases, in Rule \ref{rule:weight} and Rule \ref{rule:bias}. The definition of these uses the inverse of the $tensor$ and $matrix$ constructors, as discussed in our Z assumptions, and each is indexed by $l$, $n$, and $i$ for the weight values. 

This concludes our description of our semantic rules, that defines any ANN component in RoboChart in \Circus, next, we conclude this section with final considerations.

\section{Semantic Rules (with parameterised channels)} 

This is the semantics with parameterised channels and partial channel instnatiatons. Also, we define weights and biases as a function in the target language, that has a trivial definition based on the sequences from RoboChart. 

note: The text is not updated for these semantic rules.

This section presents our denotational semantic rules using a meta-language for ANN components in RoboChart. We use semantic brackets to denote that each \RC{ANNController} is given a precise meaning in \Circus, meaning these semantics are compositional. This matches with the RoboStar style of verification and proof. 

We define our ANN components as networks of processes. This means that we do not need state in \Circus, but our semantics in \Circus \  and not CSP broadens what we can prove about ANN components for several reasons. First, we want to create an integrated model for proofs of the complete software system in RoboChart, and we are defining the semantics of RoboChart in \Circus. Secondly, in our approach, we wish to soundly replace a controller with an ANN component, specifying the controller in \Circus \ allows the controller to have stateful behaviour, which we can use to generate verification conditions on our ANN component in our approach to proof. Section \ref{sec:verification} discusses more details of our verification approach. Finally, due to the encoding of \Circus \ and CSP. We encode our \Circus \ models in Isabelle; we do not have an automated encoding scheme or strategy for CSP processes in EMF. The UTP reactive contract theory can capture the semantics of CSP, but we do not have an automated encoding tool; it is only automated encoding in CSPM, which only permits discrete models. 

We enable two forms of automated reasoning with these semantics: first, an automated translation to Isabelle, which enables us to prove theorems about the complete behaviour of an ANN component using a predicate model; second, reasoning in CSPM enabled in RoboTool using cyclic controller semantics to verify structural properties of RoboChart models. These simplified CSP semantics are a refinement of this cyclic process pattern, so we can soundly use these semantics to establish general but limited properties about the behaviour of ANN components. 

We specify our translation rules as a set of functions from RoboChart types to \Circus \ types. We use the \Circus \ types as presented in Appendix A of Oliveira's thesis \cite{Oli06} in BNF form. We require a very limited subset of types of RoboChart to specify our translation rules, and we specify these in Z, displayed in Figure \ref{fig:robochart-ann-types}. These are to establish a clear basis for the translation rules in our meta-language, and to define some useful shorthand to simplify the description of these rules. 

\begin{figure}
    \begin{zed}
	[ Context ] \\
	ActivationFunction ::= RELU | LINEAR | NOTSPECIFIED \\
	Value == \arithmos \\
	SeqExp ::= null\_ seq | list \ldata \seq \nat \rdata | matrix \ldata \seq \seq Value \rdata | tensor \ldata \seq \seq \seq Value \rdata
\end{zed}

\begin{schema}{ANNParameters}
\\
 layerstructure : SeqExp \\
 weights : SeqExp \\
 biases : SeqExp \\
 activationfunction : ActivationFunction \\
 inputContext : Context \\
 outputContext : Context \\
\end{schema}

\begin{schema}{ANN} 
   annparameters : ANNParameters
\end{schema} 

\begin{schema}{ANNController}
\\
ANN
\end{schema}
    \caption{RoboChart types for ANN translations, specified in Z.} 
    \label{fig:robochart-ann-types}
\end{figure}

First, declare the type $Context$ to represent the RoboChart \RC{Context} class, we do not require any information from this type apart from its declaration for our translation rules. We also declare this type to link to the $allEvents$ functions, as defined in the RoboChart reference manual \cite{RoboChart}. Next, we define \RC{ActivationFunction} using a simple free type in Z, with three constants. We define $Value$ as a type synonym for the type of values communicated by our abstract ANNs, known as the arithmos type. We consider the real numbers functionally, but we define a general form of ANN, which can operate on any arithmetic type. We define $SeqExp$, a type in RoboChart representing a sequence expression, using three constructors: $list$, $matrix$, and $tensor$. We define $list$ as a sequence of natural numbers for convience, as we only require natural number sequences for our ANN components. We use $matrix$ and $tensor$ as shorthand for double or triple nested sequences, as we capture matrices and (3D) tensors using nested sequences. Defining $SeqExp$ this way enables a more convenient representation of our translation rules. 

We capture the types of our ANN classes using simple Z schemas, we use this to enable syntax matching with our EMF models defining the RoboChart metamodel. First, $ANNParameters$ is a schema with a declared variable for each element of \RC{ANNParameters}. Finally, we define $ANN$ and $ANNController$ as schemas to support syntax close to the RoboChart metamodel. 

We can now define our syntax translation rules  using a meta-language using these types. In this language, elements of the meta-langauge itself are underlined and elements of the target language, \Circus \, are in italics. The elements of the meta-language are evaluated by further calls to rules or resolving variables of the meta-language into the target language, while elements of the target target can be seen as constants elements of the functions. We have implemented these functions using the Epsilon EOL language for model-to-model transformations \footnote{\url{eclipse.dev/epsilon/doc/eol/}}.

\begin{TRule}[Semantics of ANNs \\]{ \meta{\lsem c : ANNController \rsem_{
    \mathscr{ANN}} : Program =}}
  \begin{array}[t]{l}
    \meta{ANNChannelDecl(c)} \\%
    \meta{ANNConstants(c)} \\%
    \meta{ANNProc(c)} \\%
  \end{array} \\%
  \label{rule:ANNsemantics}
\end{TRule} 

We begin with Rule \ref{rule:ANNsemantics}, our top-level rule. This rule uses the semantic brackets and takes any $ANNController$, and returns semantics that may be any number of \Circus \ paragraphs, such as channel declarations, channel set declarations, and process declarations. We represent this using the $Program$ type, which is the top-level element of the \Circus \ AST. Here, this rule calls two further meta-language rules: $ANNChannelDecl$ and $ANNProc$. 


\begin{TRule}[Function ANNChannelDecl \\]{\meta{ANNChannelDecl(c : ANNController) : Program =}}
  \begin{array}[t]{l}
    \circchannel layerRes : \nat \cross \nat \cross Value \\%
    \circchannel nodeOut : \nat \cross \nat \cross \nat \cross Value \\%
    \circchannel terminate \\%
    \circchannelset ANNHiddenEvts == \lchanset \meta{hiddenEvts} \rchanset
  \end{array} \\%
  
  \meta{\bf where} \\%
  \begin{array}[t]{l}
    \meta{hiddenEvts = \{ l, n : \nat | (0 < l < layerNo) \land n \in 1 \upto LStructure(c, l) \spot layerRes.l.n \}} \\%
  \end{array} 
  \label{rule:annchanneldecl}
\end{TRule} 

Rule \ref{rule:annchanneldecl} is the top-level rule for defining the channels required by our semantics. We first call the meta-language rule $ANNChannels$, which represents the variable channels, that is, the channels that are specific to each ANN component. We declare one channel, $terminate$, that can be seen as a constant channel in our meta-language function, and represents termination. An ANN component can never choose to engage on this event itself, but it needs to be able to engage on it to synchronise with other RoboChart components. We also declare a channel set $ANNHiddenEvts$, representing all the channels that should be hidden in the complete behaviour of our semantics. It is defined by a variable in our meta-language: $hiddenEvts$. 

We define the variable $hiddenEvts$ using a set comprehension expression. This variable defines all channels associated with the hidden layers, that should be hidden in the overall behaviour of the process. We define this variable using a set comprehension expression, using local variables $l$ representing the layer and $n$ representing the node index. We limit the values $l$ and $n$ can take using the predicate part of this set comprehension expression, and the expression we form is defining our channel using the $layerRes$ meta-language rule, which returns a channel declaration. Using this, we obtain all the channels that should be hidden in our semantics. 

We define the variable $lastLayerS$, the size of the last layer, using the inverse of the $list$ constructor of our free type for sequence expressions, in Figure \ref{fig:robochart-ann-types}. We use this to obtain a sequence of natural numbers, that we then obtain the last element of using the $last$ function, itself defined in the Z mathematical toolkit. 

Rule \ref{rule:annchannels} defines the function that represents all the variable channels in our semantics as a recursive meta-language function. This function has three parameters: $c$, an $ANNController$, $l$, representing the layer of ANN component we are considering, and $n$, the node index of the ANN component. This function computes all channels by proceeding backwards from the output layer and the last node, back to the first node in the input layer. 

The base case is when $l$ is $0$ and $n$ is $1$, this is because we use $0$ to represent the input layer, which has defined channels but no behaviour. We capture the input layer in a similar way to an interface. We use one-based indexed for the nodes, as we do not have the concept of an input node. In the base case, we return just the channel $layerRes.0.1$, as this is the channel associated with the first node of the input layer. 

In the recursive case, we first define the channel representing the layer output of the current layer and node, $layerRes(l,n)$. Then, if we not considering the channels of an input layer, that is if $l \neq 0$, then we need to consider the channels of the node itself, which we define through the meta-language function $NodeOutChannels$, which operates in a similar way to this function. The recursive call changes if we are calling the previous node in this layer, $n > 1$, or if we call the previous layer, when $n = 0$. 


Rule \ref{rule:nodeoutchannels} functions in a very similar way to Rule \ref{rule:annchannels}. We need these channels to enable intra-node communication between the processes representing the input to this node, and the collator of this node. Here, $i$ represents the input index, that is, the size of the previous layers output, used to transmit the results from the previous layer to the next layer. 

\begin{TRule}[Function ANNConstants \\]{\meta{ANNConstants(c : ANNController) : Par =}}
  \begin{array}[t]{l}
      weights : \seq \seq \seq Value; \\%
      biases : \seq \seq Value; \\%
      relu : Value \fun Value %
  \end{array} \\%
  \where %
  \begin{array}[t]{l}
      weights = \meta{((tensor \inv)~c.annparameters.weights)} \land \\%
      biases = \meta{((matrix \inv)~c.annparameters.biases)~l~n} \land \\%
      \forall x : Value @ \\%
      \t1 %
      \begin{array}[t]{l}
        (x < 0 \implies (x,0) \in relu) \land \\% 
        (x \geq 0 \implies (x,x) \in relu)
      \end{array}
   \end{array} \\%
  \label{rule:annconstants}
\end{TRule} 

\begin{TRule}[Function ANNProc \\]{\meta{ANNProc(c : ANNController) : ProcDecl =}}
  \begin{array}[t]{l} 
    \circprocess \meta{c.name} \circdef \\%
    \t1 %
    \begin{array}[t]{l}
      \circbegin \\%
      Collator \circdef l, n, i : \nat; sum : Value \circspot \\%
      \begin{array}[t]{l}
          \begin{array}[t]{l}
              \lcircguard i = 0 \rcircguard \circguard layerRes.l.n~!(relu(sum + ( biases(l)(n)))) \then \Skip \\% \extchoice \\%
            \lcircguard i > 0 \rcircguard \circguard nodeOut.l.n.i~?x \then Collator(l, n, (i-1), (sum+x))
        \end{array}
      \end{array} \\%
      NodeIn \circdef l, n, i : \nat \circspot \\%
      \begin{array}[t]{l}
        \begin{array}[t]{l} 
             layerRes.(l-1).i~?x \then nodeOut.l.n.i~!(x * (weights(l)(n)(i))) \then \Skip 
        \end{array} \\%
      \end{array} \\%
      Node \circdef l, n, inpSize : \nat \circspot \\%
      \begin{array}[t]{l}
        \begin{array}[t]{l} 
          ((\Interleave i: 1 \upto inpSize \circspot NodeIn(l, n, i)) \\%
          \lpar | \lchanset nodeOut.l.n \rchanset | \rpar \\%
          Collator(l, n, inpSize, 0) \circhide \lchanset nodeOut.l.n \rchanset )
        \end{array}
      \end{array} \\%
      HiddenLayer \circdef l, s, inpSize : \nat \circspot \\%
      \begin{array}[t]{l} 
        \begin{array}[t]{l} 
         (\lpar \lchanset layerRes.(l-1) \rchanset \rpar i : 1 \upto s \circspot Node(l, i, inpSize))          
       \end{array} \\%
      \end{array} \\%
      HiddenLayers \circdef \meta{HiddenLayers(c, layerNo(c) - 1)} \\%
      OutputLayer \circdef \\%
        \begin{array}[t]{l} 
          \lpar \lchanset layerRes.\meta{(layerNo(c)-1)} \rchanset \rpar i : 1 \upto \meta{LStructure(c, layerNo(c))} \circspot \\%
         \t1 Node(l, i, \meta{LStructure(c, layerNo(c)-1)}))
        \end{array} \\%
      ANN \circdef %
      \begin{array}[t]{l}
        ((HiddenLayers \lpar | layerRes.\meta{layerNo(c)-1)} | \rpar OutputLayer) \\%
         \circhide ANNHiddenEvts) \circseq ANN \\% 
      \end{array} \\%
      
      ANNRenamed \circdef \meta{ANNRenamed(c)} \\%
      
      \circspot ANNRenamed \\%
      
      \circend \\
    \end{array}
    
  \end{array} \\%
        
  \label{rule:annproc}
\end{TRule} 

We now consider the rules which define the process paragraph for our ANN model. We start with the top-level process rule $ANNProc$ in Rule \ref{rule:annproc}. In a paragraph, we define multiple actions, similar to CSP processes, then a main action after the syntax $\circspot$ which represents the behaviour of this process. We name our process using the $name$ attribute of our $ANNController$. As an \RC{ANNController} is a \RC{NamedElement} in RoboChart, we have access to this attribute. In our first four actions, $Collator$, $NodeIn$, $Node$, and $HiddenLayer$, we have fixed local variables, represented by non-underlined syntax in our meta-language, followed by a meta-language function representing the ANN-specific definition of this action. 

For the last four definitions, we do not use local variable definitions as the behaviour of these actions represents the overall structure of an ANN pattern, their behaviour is less influenced by the hyperparameters of an ANN model. $HiddenLayers$ and $OutputLayer$ are each defined by a meta-language function capturing the composition of the hidden layers and the behaviour of just the output layer in isolation. The action $ANN$ defines the main behaviour of an ANN component: it is the repeating parallel composition of $HiddenLayers$ and $OutputLayer$, with $ANNHiddenEvts$ hidden. We capture the synchronisation set of this parallel composition using the $IndexedLayerRes(layerNo(c) - 1)$ meta-function, that returns all channels in the penultimate layer, where $layerNo(c)$ is the number of layers in the ANN model. We use the function $ANNRenamed$ to capture the contextual renaming of our ANN component to events of the RoboChart model. Finally, our main action is defined as $ANNRenamed$, which captures the behaviour of our ANN semantics. 

Rule \ref{rule:collator} defines the rule for the behaviour of the $Collator$ action, represented using the $CSPAction$ AST element. This rule defines, at its core, a recursive action that sums all values communicated by the channel $nodeOut$, and then communicates these results through the $layerRes(l,n)$ channel as its base case, with the $bias$ from the $bias$ meta-function. This recursive action, in the target language, is also called \textit{Collator}, and is shown here as a constant element in the recursive case in this rule. The constructs surrounding this behaviour is a distributed external choice over $l$, the layer size, $n$, the node size, and $i$, the input size, the size of the previous layer. Each is guarded by checking that the target language variable is equal to the meta-language variable, meaning only one behaviour is possible at every parameter instantiation. All of this structure is only needed to support the creation of distinct channels for every layer, node, and input size, as is required in \Circus. 

Rule \ref{rule:nodein} defines the behaviour of the $NodeIn$ action: to receive the value communicated through the channel $layerRes(l-1, i)$, then to transmit a modified version of this using the channel $nodeOut(l,n,i)$. It is a buffer which applies the appropriate weight to the value communicated by the previous layer's event, it is a modifying buffer. In our \Circus \ model of an ANN's behaviour, each $NodeIn$ action is used to transmit the result from the previous layer onto the current layer. We use the $nodeOut$ channel as an intermediate communication channel, that every $Collator$ process synchronises on to obtain the weighted output of the previous layer. We use a surrounding structure identical to that of Rule \ref{rule:collator}, to support distinct channel naming. 
 
Similar to Rule \ref{rule:nodein} and Rule \ref{rule:collator}, Rule \ref{rule:node} uses the external choice and guarding considering the $l$ and $n$ indices to support the distinct channel naming. This rule is defined by the parallel composition of two actions. First, a distributed interleaving of all $NodeIn$ processes indexed by the layer, node, and considering every input, from $inpSize$. Here, $inpSize$ is one of the parameters for $Node$ in the target language, see Rule \ref{rule:annproc}. This entire action is defined in the target language, as its definition does not change based on the parameters of $Node$. The second action is the $Collator$ action to receive all communications from the $NodeIn$ interleavings and transmit the result on the $layerRes$ channel. To define the channels we synchronise on, we use $IndexedNodeOut(l,n)$, a meta-function that returns all $nodeOut$ channels associated with this node: $nodeOut.l.n.i$ where $i$ is the previous layers size. We use the channel set operator, denoted using $\lchanset \rchanset$, of these channels to define all events that can be communicated using this set of channels. Finally, we hide this synchronisation set from the resultant process, so the internal communication channel $nodeOut$ is not visible in the $Node$ action. 

Rule \ref{rule:hiddenlayer} defines the behaviour of our $HiddenLayer$ action. We define this using a replicated parallel, synchronised on the channel set of all $layerRes$ channels associated with this layer, captured by the meta-function $IndexedLayerRes(l-1)$. A layer in an ANN model synchronises on the previous layer's result ($l-1$), because these events are shared across every node in this layer. Each node process engages with a single event indexed by $layerRes.l$ to communicate its output, these communications do not need to be synchronised as each node has a unique output channel. Using this rule, Rule \ref{rule:hiddenlayers} defines the behaviour of an arbitrary number of hidden numbers as a recursive composition of parallel actions. 

\begin{TRule}[Function HiddenLayers \\]{\meta{HiddenLayers(c : ANNController, l : \nat) : CSPAction =}}
  \begin{array}[t]{l} 
      \meta{\bf{if(l == 1)}} \\%
      \meta{\bf{then}} \\%
      \t1 %
      \begin{array}[t]{l}
        (HiddenLayer(\meta{l}, \meta{LStructure(c, l)}, \meta{LStructure(c, l-1)}))
      \end{array} \\%
      \meta{\bf{else}} \\%
      \t1 %
      \begin{array}[t]{l}
        (\meta{HiddenLayers(c, (l-1))} \\%
        \lpar | \lchanset layerRes.\meta{(l-1)} \rchanset | \rpar \\%
        HiddenLayer(\meta{l}, \meta{LStructure(c, l)}, \meta{LStructure(c, l-1)})
      \end{array}
  \end{array} \\%
  \label{rule:hiddenlayers}
\end{TRule} 

Rule \ref{rule:outputlayer} defines the output layer for our ANN model. This layer can be seen as the definition of a layer whose parameters are constant, using the meta-function $layerNo(c)$ as a constant to define the shape of our output layer. We separate the hidden layers and the output layer of an ANN model to capture the common pattern used when designing an ANN model: there is a fixed shape to the hidden layers, for example $6$ layers with $50$ nodes in each \cite{CAV2019}, and the shape of both the input and output layers are distinct from this shape, for example $5$ and $5$. Moreover, the shape of the input and output layers are defined by the context where an ANN model is used, but the shape of the hidden layers is defined by the complexity of the desired relation, which is independent of context and is a training consideration. Finally, output layers often have probabilistic interpretations, and their results are subject to additional external analysis, that we can support easier with a separate defined output layer action in \Circus. 

\begin{TRule}[Function ANNRenamed \\]{\meta{ANNRenamed(c : ANNController) : CSPAction) = }} 
  \begin{array}[t]{l}
    (ANN) %
        \begin{array}[t]{l}
          \lcircrename \meta{orderedLayerRes} := \meta{eventList} \rcircrename %
          \circinterrupt ~ terminate \then \Skip \\%
        \end{array} \\%
  \end{array} \\%
  
  \meta{\bf{where}}  \\%
  
  \begin{array}[t]{l}
    \meta{orderedLayerRes =} \\%
    \t1 \meta{order(\{ l : \{0, layerNo(c)\} ; n : 1 \upto LStructure(c, l)} \meta{ @ layerRes.l.n\})} \\%
    \meta{eventList = order(allEvents(c.annparameters.inputContext)) \cat }\\%
    \t1 \meta{order(allEvents(c.annparameters.outputContext)) }
  \end{array}
  \label{rule:annrenamed}
\end{TRule}

We present the last of our functional rules, that defines the process presented in Rule \ref{rule:annproc}, in Rule \ref{rule:annrenamed}. This rule defines the target language action $ANN$, see Rule \ref{rule:annproc}, where a renaming is applied, is interrupted ($\circinterrupt$) by the action $terminate \then \Skip$. This is to support if the other components in the system terminate, represented by the event $terminate$, then the ANN component should terminate, behave as $\Skip$, as well. The renaming operation is to rename the $layerRes$ channels to the contextual channels as defined by the user in the $inputContext$ and $outputContext$ in RoboChart. We define that all events in $orderedLayerRes$ are renamed to the events from $eventList$, the contextual events. Here, we define $orderedLayerRes$ by a set comprehension expression $layerRes(l,n)$, where $l$ is either $0$ or $layerNo$ (input or output layer), and $n$ is all nodes in these layers. We define $eventList$ to be all events, using the $allEvents$ meta-function \cite{RoboChart}, of the input context concatenated with the events of the output context. The $order$ function is then applied to both of these sets, here we assume this function takes a set and creates a sequence ordering this set, we can implement this function in EMF as the references of an object is defined using lists. In other words, in our implementation is based on the order that the events appear in the textual language for RoboChart, so, in our current implementation, the ordering of events is an important consideration for ANN components. 

\begin{TRule}[Function LStructure \\]{\meta{LStructure(c : ANNController, i : \nat) : \nat =}}
  \begin{array}[t]{l}
    \meta{\bf{if(i == 0)}} \\
    \meta{\bf{then}} \\
    \t1 %
	  \meta{ \# allEvents(c.annparameters.inputContext) } \\
    \meta{\bf{else}} \\
    \t1 %
      \meta{((list \inv) c.annparameters.layerstructure)~i}
  \end{array} 
  \label{rule:lstructure}
\end{TRule} 

\begin{TRule}[Function layerNo (number of layers) \\]{\meta{layerNo(c : ANNController) : \nat =}}
  \begin{array}[t]{l}
    \meta{\# ((list \inv)~c.annparameters.layerstructure)}
  \end{array} 
  \label{rule:layerno}
\end{TRule} 

The explanation of the rules that define the functional behaviour of the semantics for our ANN components now complete. Rules \ref{rule:layerres} through to our final rule \ref{rule:indexednodeout} present helper functions for values, channels, and channel sets used to define the behaviour of our rules. These rules also enable a cleaner syntax for the presentation of the functional rules. 

Rule \ref{rule:lstructure} and Rule \ref{rule:layerno} define constants about our ANN model. We define $LStructure$ to be the size of each layer, using the inverse of the $list$ free type constructor indexed by $i$, and the number of events in the input context if we are referring to the input layer ($l = 0$). The function $layerNo$ defines the number of layers in our model, which is simply the size of the $layerstructure$ sequence in our $ANNController$. 

Next, we discuss our helper rules that refer to channels, a $CSExp$ in \Circus. Our basic channels are defined in Rule \ref{rule:layerres} and Rule \ref{rule:nodeout}, defining $layerRes$ with $l$ and $n$, and $nodeOut$ indexed with $l$, $n$, and $i$. We define the indexed versions of these channels in Rules \ref{rule:indexedlayerres} and \ref{rule:indexednodeout}. The indexed versions take all but the last parameter, and return the channel set containing of all channels constructed using the $l$ parameter for $layerRes$, and the $l$ and $n$ parameter for $nodeOut$. Finally, for the $nodeOut$ channel only, we define $AllNodeOut$ in Rule \ref{rule:allnodeout}. This function is a shorthand for the channel set of $nodeOut$ channels used by the $ANNController$ $c$. To describe this, we define the number of layers, for $l$, by $layerNo$, the size of each layer, for $n$, by $LStructure(l)$, and the size of the previous layer, for the index $i$, set by $LStructure(l-1)$. 




Finally, we define our trained parameters, the weights and biases, in Rule \ref{rule:weight} and Rule \ref{rule:bias}. The definition of these using the inverse of the $tensor$ and $matrix$ constructors, as discussed in our Z assumptions, and each are indexed by $l$, $n$, and $i$ for the weight values. 

This concludes our description of our semantic rules, that defines any ANN component in RoboChart in \Circus, next, we conclude this section with final considerations.

\section{AnglePIDANN Circus Program}

\subsection{Preliminary Material}

\begin{figure}[t]
\begin{zed}
  Value ~~==~~ \real
\end{zed}

\begin{circus}
  \circchannel layerRes01: Value \\
  \circchannel layerRes02: Value \\
  \circchannel layerRes11: Value \\
  \circchannel layerRes21: Value \\
  \circchannel nodeOut111: Value \\
  \circchannel nodeOut112: Value \\
  \circchannel nodeOut211: Value \\
  \circchannel\ terminate \\
\end{circus}

\begin{circus}
  \circchannel adiff\_in : \real \\
  \circchannel anewError\_in : \real \\
  \circchannel angleOutputE\_out : \real \\
\end{circus}

\begin{circus}
  \circchannelset ANNHiddenEvts == \lchanset layerRes11 \rchanset
\end{circus}

DON'T NEED THIS, for the meta-language. 
\begin{axdef}
  relu : Value \fun Value %
  \where %
  \forall x : Value @ \\%
  \t1 %
  (x < 0 \implies (x,0) \in relu) \land \\% 
  (x \geq 0 \implies (x,x) \in relu)
\end{axdef}

  \caption{The preliminary Circus paragraphs, for the $AnglePIDANN$ example. }
  \label{anglepidann-example-preliminaries}
\end{figure} 

\subsection{Process Definition} 

\begin{figure}[p]
  
\begin{circus}
  \circprocess\ AnglePIDANN \circdef \\%
  \t1 %
    \begin{array}[t]{l}
      \circbegin \\%
      Collator \circdef l, n, i : \nat; sum : Value \circspot \\%
      \t1 %
      \begin{array}[t]{l} 
        \lcircguard l = 1 \land n = 1 \land i = 0 \rcircguard \circguard layerRes11~!(relu(sum + ( 0 \decimalpoint 125424 ))) \then \Skip \\
        \extchoice~ \lcircguard l = 1 \land n = 1 \land i = 1 \rcircguard \circguard nodeOut111~?x  \then Collator(l, n, (i-1), (sum+x)) \\%
        \extchoice~ \lcircguard l = 1 \land n = 1 \land i = 2 \rcircguard \circguard nodeOut112~?x  \then Collator(l, n, (i-1), (sum+x)) \\%
        \extchoice~ \lcircguard l = 2 \land n = 1 \land i = 0 \rcircguard \circguard layerRes21~!(relu(sum + ( \negate 0 \decimalpoint 107753 ))) \then \Skip \\%
        \extchoice~ \lcircguard l = 2 \land n = 1 \land i = 1 \rcircguard \circguard nodeOut211~?x \then Collator(l, n, (i-1), (sum+x)) \\%
      \end{array} \\%
      
      NodeIn \circdef l, n, i : \nat \circspot \\%
      \t1 %
      \begin{array}[t]{l}
        \lcircguard l = 1 \land n = 1 \land i = 1 \rcircguard \circguard layerRes01~?x \then nodeOut111~!(x * (1 \decimalpoint 22838)) \then  \Skip \\%
        \extchoice~ \lcircguard l = 1 \land n = 1 \land i = 2 \rcircguard \circguard layerRes02~?x \then nodeOut112~!(x * (0 \decimalpoint 132874)) \then \Skip \\%
        \extchoice~ \lcircguard l = 2 \land n = 1 \land i = 1 \rcircguard \circguard layerRes11~?x \then nodeOut211~!(x * (0 \decimalpoint 744636)) \then \Skip \\%
      \end{array} \\%
      
    
      Node \circdef l, n, inpSize : \nat \circspot \\%
      \t1 %
      \begin{array}[t]{l}
        \lcircguard l = 1 \land n = 1 \rcircguard \circguard %
        \begin{array}[t]{l}
          ((\Interleave i: 1 \upto inpSize \circspot NodeIn(l, n, i)) \\%
          \lpar | \lchanset nodeOut111, nodeOut112 \rchanset | \rpar \\%
          Collator(l, n, inpSize, 0) \circhide \lchanset nodeOut111, nodeOut112 \rchanset )
        \end{array} \\%
        \extchoice~\lcircguard l = 2 \land n = 1 \rcircguard \circguard %
        \begin{array}[t]{l}
          ((\Interleave i: 1 \upto inpSize \circspot NodeIn(l, n, i)) \\%
          \lpar | \lchanset nodeOut211 \rchanset | \rpar \\%
          Collator(l, n, inpSize, 0) \circhide \lchanset nodeOut211 \rchanset) \\%
        \end{array}
      \end{array} \\%
        
      HiddenLayer \circdef l, s, inpSize : \nat \circspot \\%
      \t1 %
      (\lpar \lchanset layerRes01, layerRes02 \rchanset \rpar i : 1 \upto s \circspot Node(l, i, inpSize)) \\%
        
      HiddenLayers \circdef \\%
      \t1 %
      HiddenLayer(1, 1, 2) \\%
      
      OutputLayer \circdef \\%
      \t1 %
      \lpar \lchanset layerRes11 \rchanset \rpar i : 1 \upto 1 \circspot Node(2, i, 1) \\%
      
      ANN \circdef \\%
      \t1 %
      ((HiddenLayers \lpar | \lchanset layerRes11 \rchanset | \rpar OutputLayer) \circhide ANNHiddenEvts) \circseq ANN \\%
      
      ANNRenamed \circdef \\%
      \t1 %
      (ANN) %
        \begin{array}[t]{l}
          \lcircrename layerRes01, layerRes02, layerRes21 := \\%
          \t1 anewError\_in, adiff\_in, angleOutputE\_out \rcircrename \\%
          \circinterrupt~ terminate \then \Skip \\%
        \end{array} \\%
        \circspot ANNRenamed \\%
        \circend
      \end{array} \\%
\end{circus}    

  \caption{$AnglePIDANN$ example, in Circus} 
  \label{fig:anglepidann_circus_example}
\end{figure} 




\section{Notes}

Differences to the CSP semantics, where the Circus actions representing the CSP processes differ. We have proved, in FDR, for the $AnglePIDANN$ and $AnglePIDANN2$ examples, and for a binarised version of $AnglePIDANN$, that our Circus semantics are equivalent, in the traces model, to the CSP semantics.

\begin{itemize} 
\item channels are renamed, no longer use indexed channels, the indexed channel abstraction. There are multiple cases, on each process, with a guard and each process is chained together by external choice, such that only one process should not evaluate to $STOP$, the rest are $STOP \extchoice P$, which evaluates to $P$. 
\item ANNHiddenEvts defined constructively, not all those events without the inputs and outputs. 

\item We are not hiding all of node out, like we do in the original, because unnnecessary, and in the meta-language, we have to define it anyway, $\lchanset nodeOut.1.1 \rchanset$ we have to define anyway, so its cleaner, to define we synchronise on that, then hide just that. 
\item parallel synchronisation, without the variable sets, needs $|$ characters, from the circus guide, it should not, but it does for us: $\lpar | \lchanset layerRes11 \rchanset | \rpar $. $\lpar layerRes11 \rpar$ does not compile, when it says this is valid syntax. 

\item No longer using replicated alphabetsied parallel in $HiddenLayers$, we are now using multiple generalised parallel in $HiddenLayers$. 
\end{itemize} 


Issues or Questions: 
\begin{itemize} 
  \item Stateless, so we omit the state reserved word, allowed in CZT, but Marcel's BNF seem to imply it is always required. 
  \item In this document, I am using the Circus latex style, not the csp or CZT, so we are using circinterrupt instead of interrupt for CSP interrupt, but both produce the same symbol. 
  \item We use binarised parameters, and the $sign$ activation function instead of $relu$, for validation of our Circus programs. 
  \item We use the roboworld 2d toolkit, we only really need the real type, and the decimal point definition, in CZT, but we do need this declaration, otherwise the process would be very ugly, but this is required as well as the standard circus toolkit, to write the circus programs in CZT. 
\end{itemize} 

\section{ANN Circus semantics in the Circus metamodel}

\subsection{Important Classes in metamodel} 
\begin{itemize}
\item Term (abstract class, represents a term). 
\item Para -> Term (abstract class, represents a paragraph). 
\item Types of Para: AxPara, ActionPara, ChannelPara, ChannelSetPara, ConjPara, FreePara, GivenPara, NameSetPara, ProcessPara, etc.  
\item Sect, is a Term, abstract class. 
\item Concrete subclasses of Sect, ZSect, that is it, just ZSect. 
\item ZSect -> Sect (Z section, has name: EString, paraList: ParaList, parents: Parent).
\item ParaList, abstract class, list of paragraphs, ZParaList, paras, ZParaList is a concrete type of ParaList. 
\item ZParaList, concrete ParaList, list of, paras: Para. 
\item Para, 
\item ConstDecl, has name: ZName, and expr: Expr
\item Expresssions, Expr, types of expression: 
\item Expr, is a term, an abstract class. 
\item Concrete instantiations: BasicChannelSetExpr, BindExpr, CondExpr, NumExpr, RefExpr, SigmaExpr, SchExpr, RefExpr, 
\item RefExpr
\item SchTExt, ZSchText, schema text. 
\item ZName, is word, id, operatorName, strokesLsit, strokesList, list of Z strokes used.
\item word is the name of ZName, an EString. 
\item id, 8571, just a number in CZT. real is 

\end{itemize} 
Notes from CZT API, https://czt.sourceforge.net/corejava/corejava-z/apidocs/index.html

\begin{itemize}
\item Stroke, is a Term, an abstract Stroke, 
\item Stroke, ?, 

\end{itemize} 


This is the AST, 
\subsection{Channel declarations, and the ReLU declaration}
The CZT Circus AST, representation of our example, AST for our example: 

\begin{itemize}
  \item Horizontal Definition Paragraph ( 
  \item Schema text "Value"
  \item List of declarations "Value"
  \item Constant declaration "Value" (has a name and a reference expression) [ConstDecl in EMF]
  \item name, then a reference expression, has name, 
  \item reference expression "real", [RefExpr in EMF]
  \item reference expression, has a list of expressions, [expression list]. 
  \item 
  
\end{itemize}

Name of the constant declaration, is "Value", name of the reference expression, "real", list of expressions, no Z strokes, 

\subsection{Mechanisation Notes (EOL)} 
Mechanisation, in CZT Circus API, of the various BNF rules that we refer to, is: 

\begin{itemize}
  \item Program: multiple circus paragraphs, in CZT AST, this is: List of Paragraphs, in Tool, it is ZParaList, in Circus AST, you can put Circus and Z paragraphs in this, it is just a list of Paras, which can be either. 
   
  \item ProcessPara, is ProcDecl, potentially, defining a process paragraph. 
  \item We are using, in EOL, no c : ANNController, that is the self, that is the context of the operation, all the semantics are operations on ANNController objects. 
  \item CircusProcess, absrtact, of BasicProcess, with parameters, mainAction, ontheFlyParagraphs, paragraph lists, Axiom paragraphs, state para. State paragraph list. Them has local paragraphs. 
  \item ProcessPara, has a CircusProcess, a namelist, a name, and if it is a basic process or not. 
  \item The AnglePIDANN, overall, is a process pargraph, not a circusprocess, then the basic process, then basic process has a list of paragraphs. 
  \item Then, each is an action paragraph, each CSP process, 
  \item Treat the $Value == \real$, horizontal definition paragraph, as Part of the Toolkit, as imported in CZT. 
  \item Not using explicit paragraph lists, generating a document with just using, the actual paragraphs. They aren't grouped in the CZT file anyway, it was grouped, in the metalanguge, in the BNF, more just to show, to describe what they are. 
  \item We don't have a section, that would have a list of paragraphs, we just have the list of paragraphs, doesn't matter. Still automatically generates them. 
  \item Do we need the explicit import? and Section header? in the M2M? Does that exist yet? 
  \item Each channel is declared in its own channel paragraph, one channel paragraph per channel declaration.
  \item layerRes and nodeOut are declared as STRINGS, not channels, as easier just to get the names, then call the functions to get the channel paragraphs, and declarations, from other places. 
  \item Process Paragraph, top level, name "AnglePIDANN". Basic Process, then has list of paragraphs, aciton paragraphs, then horizontal definition paragraph. 
  \item Axiomatic description paragraph relu. That can be in the toolkit as well. 
  \item createProcessPara, needs a Z!CircusProcess, sets the circusprocess, to that, name: and is BasicProcess, isBasicProc, true, 
  \item createParallelProcess, name, left, and right. $cs_name$, reference to channelset name, creates a channel set, with the reference expression, of $cs_name$. 
  \item Can also, createParallelProcess, with $channel_names$ as a set, left, and right. 
  \item createCallProcess, call expression, 
  \item createBasicProcess, mainActionName, sets the main action. 
  \item give a sequence of 
  \item the basic process, paragraph lists, is the list. 
  \item paragraphs, is the sequence. 
  \item From a Z!Para, can create a basic process, with a main action name, this is a Z!BasicProcess. 
  \item create action, createPrefixingAction, createAction1. 
  \item mechanisation of process, top level: Process Paragraph, then has a name, the process paragraphs name is "AnglePIDANN", the name of the RC component. That is just the anglepidann.name, it does work. 
  \item then has a basic process, then in this basic process, has a pargraph list, then it has ACTION PARAGRAPHS, then a main action, then a horizontal definition paragraph, default? 
  \item Fang's mechanisation, find process paragraphs, then basic processes, and action paragraphs. 
  \item createProcessPara(name: String, isBasicProc: Boolean), context of a CircusProcess, creates a ProcessPara. sets the name to name, and isBasicProcess, 
  \item createCallProcess(), creates a Ref Expression, a process that calls a reference, a refernece expression. 
  \item createParallelProcess, 
  \item createActionPara(), Circus Action, returns an ActionPara, takes a CircusAction, creates an action paragraph. with the circus action = to self. 
  \item createCircusAction, 
  \item how the events are represented in memory, 
  \item Events are not ordered, in EMF RoboChart models, based on when users, I saw that somewhere they were? But not in EOL itself, not ordered. 
  \item ordered is FALSE on events, saw in representation, in parts of xtext code, not in EMF, not an ordered. 
  \item ORDERING WORKS, NOT SURE WHY, SAYS UNORDERED, IN EMF, BUT IT SEEMS TO WORK, EVEN IN INTERFACES. MULTIPLE, ONE AT A TIME, SURELY. 
\end{itemize} 

\appendix 

\section{AnglePIDANN2 Circus Program}

Used to make an example with more than one hidden layer, and more than one layer per node. 

\begin{figure}[p] 
\begin{circus}
  \circprocess\ AnglePIDANN2 \circdef \ \circbegin \\%
      Collator \circdef l, n, i : \nat; sum : Value \circspot \\
      \lcircguard l = 1 \land n = 1 \land i = 0 \rcircguard \circguard layerRes11~!(sign(sum + ( 0 ))) \then \Skip \\%
      \extchoice
      \lcircguard l = 1 \land n = 1 \land i = 1 \rcircguard \circguard nodeOut111~?x  \then Collator(l, n, (i-1), (sum+x)) \\%
      \extchoice
      \lcircguard l = 1 \land n = 1 \land i = 2 \rcircguard \circguard nodeOut112~?x  \then Collator(l, n, (i-1), (sum+x)) \\%
      \extchoice
      \lcircguard l = 1 \land n = 2 \land i = 0 \rcircguard \circguard layerRes12~!(sign(sum + ( 0 ))) \then \Skip \\%
      \extchoice
      \lcircguard l = 1 \land n = 2 \land i = 1 \rcircguard \circguard nodeOut121~?x  \then Collator(l, n, (i-1), (sum+x)) \\%
      \extchoice
      \lcircguard l = 1 \land n = 2 \land i = 2 \rcircguard \circguard nodeOut122~?x  \then Collator(l, n, (i-1), (sum+x)) \\%
      \extchoice
      \lcircguard l = 1 \land n = 3 \land i = 0 \rcircguard \circguard layerRes13~!(sign(sum + ( 0 ))) \then \Skip \\%
      \extchoice
      \lcircguard l = 1 \land n = 3 \land i = 1 \rcircguard \circguard nodeOut131~?x  \then Collator(l, n, (i-1), (sum+x)) \\%
      \extchoice
      \lcircguard l = 1 \land n = 3 \land i = 2 \rcircguard \circguard nodeOut132~?x  \then Collator(l, n, (i-1), (sum+x)) \\%
      \extchoice
      
      
      \lcircguard l = 2 \land n = 1 \land i = 0 \rcircguard \circguard layerRes21~!(sign(sum + ( 0 ))) \then \Skip \\%
      \extchoice
      \lcircguard l = 2 \land n = 1 \land i = 1 \rcircguard \circguard nodeOut211~?x \then Collator(l, n, (i-1), (sum+x)) \\%
      \extchoice
      \lcircguard l = 2 \land n = 1 \land i = 2 \rcircguard \circguard nodeOut212~?x \then Collator(l, n, (i-1), (sum+x)) \\%
      \extchoice
      \lcircguard l = 2 \land n = 1 \land i = 3 \rcircguard \circguard nodeOut213~?x \then Collator(l, n, (i-1), (sum+x)) \\%
      \extchoice
      
      \lcircguard l = 3 \land n = 1 \land i = 0 \rcircguard \circguard layerRes31~!(sign(sum + ( 0 ))) \then \Skip \\%
      \extchoice
      \lcircguard l = 3 \land n = 1 \land i = 1 \rcircguard \circguard nodeOut311~?x \then Collator(l, n, (i-1), (sum+x)) \\%
      \extchoice
      
      \lcircguard l = 4 \land n = 1 \land i = 0 \rcircguard \circguard layerRes41~!(sign(sum + ( 0 ))) \then \Skip \\%
      \extchoice
      \lcircguard l = 4 \land n = 1 \land i = 1 \rcircguard \circguard nodeOut411~?x \then Collator(l, n, (i-1), (sum+x)) \\%
      \extchoice
      \lcircguard l = 4 \land n = 2 \land i = 0 \rcircguard \circguard layerRes42~!(sign(sum + ( 0 ))) \then \Skip \\%
      \extchoice
      \lcircguard l = 4 \land n = 2 \land i = 1 \rcircguard \circguard nodeOut421~?x \then Collator(l, n, (i-1), (sum+x)) \\%
      
      
      NodeIn \circdef l, n, i : \nat \circspot \\%
      \lcircguard l = 1 \land n = 1 \land i = 1 \rcircguard \circguard layerRes01~?x \then nodeOut111~!(x * ( 1 )) \then \Skip \\%
      \extchoice \\%
      \lcircguard l = 1 \land n = 1 \land i = 2 \rcircguard \circguard layerRes02~?x \then nodeOut112~!(x * ( 1 )) \then \Skip \\%
      \extchoice \\%
      \lcircguard l = 1 \land n = 2 \land i = 1 \rcircguard \circguard layerRes01~?x \then nodeOut121~!(x * ( 1 )) \then \Skip \\%
      \extchoice \\%
      \lcircguard l = 1 \land n = 2 \land i = 2 \rcircguard \circguard layerRes02~?x \then nodeOut122~!(x * ( 1 )) \then \Skip \\%
      \extchoice \\%
      \lcircguard l = 1 \land n = 3 \land i = 1 \rcircguard \circguard layerRes01~?x \then nodeOut131~!(x * ( 1 )) \then \Skip \\%
      \extchoice \\%
      \lcircguard l = 1 \land n = 3 \land i = 2 \rcircguard \circguard layerRes02~?x \then nodeOut132~!(x * ( 1 )) \then \Skip \\%
      \extchoice \\%
      
      \lcircguard l = 2 \land n = 1 \land i = 1 \rcircguard \circguard layerRes11~?x \then nodeOut211~!(x * ( 1 )) \then \Skip \\%
      \extchoice
      \lcircguard l = 2 \land n = 1 \land i = 2 \rcircguard \circguard layerRes12~?x \then nodeOut212~!(x * ( 1 )) \then \Skip \\%
      \extchoice
      \lcircguard l = 2 \land n = 1 \land i = 3 \rcircguard \circguard layerRes13~?x \then nodeOut213~!(x * ( 1 )) \then \Skip \\%
      \extchoice
      
      \lcircguard l = 3 \land n = 1 \land i = 1 \rcircguard \circguard layerRes21~?x \then nodeOut311~!(x * ( 1 )) \then \Skip \\%
      \extchoice
      
      \lcircguard l = 4 \land n = 1 \land i = 1 \rcircguard \circguard layerRes31~?x \then nodeOut411~!(x * ( 1 )) \then \Skip \\%
      \extchoice
      \lcircguard l = 4 \land n = 2 \land i = 1 \rcircguard \circguard layerRes31~?x \then nodeOut421~!(x * ( 1 )) \then \Skip \\%
      
    
      Node \circdef l, n, inpSize : \nat \circspot \\%
        \lcircguard l = 1 \land n = 1 \rcircguard \circguard
        ((\Interleave i: 1 \upto inpSize \circspot NodeIn(l, n, i)) \\%
        \lpar | \lchanset nodeOut111, nodeOut112 \rchanset | \rpar \\%
        Collator(l, n, inpSize, 0) \circhide \lchanset nodeOut111, nodeOut112 \rchanset \\   
        ) \\
        \extchoice \\
        
        \lcircguard l = 1 \land n = 2 \rcircguard \circguard
        ((\Interleave i: 1 \upto inpSize \circspot NodeIn(l, n, i)) \\%
        \lpar | \lchanset nodeOut121, nodeOut122 \rchanset | \rpar \\%
        Collator(l, n, inpSize, 0) \circhide \lchanset nodeOut121, nodeOut122 \rchanset \\   
        ) \\
        \extchoice \\
        
        \lcircguard l = 1 \land n = 3 \rcircguard \circguard
        ((\Interleave i: 1 \upto inpSize \circspot NodeIn(l, n, i)) \\%
        \lpar | \lchanset nodeOut131, nodeOut132 \rchanset | \rpar \\%
        Collator(l, n, inpSize, 0) \circhide \lchanset nodeOut131, nodeOut132 \rchanset \\   
        ) \\
        \extchoice \\
        
        \lcircguard l = 2 \land n = 1 \rcircguard \circguard
        ((\Interleave i: 1 \upto inpSize \circspot NodeIn(l, n, i)) \\%
        \lpar | \lchanset nodeOut211, nodeOut212, nodeOut213 \rchanset | \rpar \\%
        Collator(l, n, inpSize, 0) \circhide \lchanset nodeOut211, nodeOut212, nodeOut213 \rchanset \\   
        ) \\
        \extchoice
        
        \lcircguard l = 3 \land n = 1 \rcircguard \circguard
        ((\Interleave i: 1 \upto inpSize \circspot NodeIn(l, n, i)) \\%
        \lpar | \lchanset nodeOut311 \rchanset | \rpar \\%
        Collator(l, n, inpSize, 0) \circhide \lchanset nodeOut311 \rchanset \\   
        ) \\
        \extchoice
        
         \lcircguard l = 4 \land n = 1 \rcircguard \circguard
        ((\Interleave i: 1 \upto inpSize \circspot NodeIn(l, n, i)) \\%
        \lpar | \lchanset nodeOut411 \rchanset | \rpar \\%
        Collator(l, n, inpSize, 0) \circhide \lchanset nodeOut411 \rchanset \\   
        ) \\
        \extchoice
        
         \lcircguard l = 4 \land n = 2 \rcircguard \circguard
        ((\Interleave i: 1 \upto inpSize \circspot NodeIn(l, n, i)) \\%
        \lpar | \lchanset nodeOut421 \rchanset | \rpar \\%
        Collator(l, n, inpSize, 0) \circhide \lchanset nodeOut421 \rchanset \\   
        ) \\
        
        
      HiddenLayer \circdef l, s, inpSize : \nat \circspot \\%
      \lcircguard l = 1 \rcircguard \circguard
      (\lpar \lchanset layerRes01, layerRes02 \rchanset \rpar i : 1 \upto s \circspot Node(l, i, inpSize)) \\%
      \extchoice 
      
      \lcircguard l = 2 \rcircguard \circguard
      (\lpar \lchanset layerRes11, layerRes12, layerRes13 \rchanset \rpar i : 1 \upto s \circspot Node(l, i, inpSize)) \\%
      \extchoice 
      
      \lcircguard l = 3 \rcircguard \circguard
      (\lpar \lchanset layerRes21 \rchanset \rpar i : 1 \upto s \circspot Node(l, i, inpSize)) \\%
      
      
      HiddenLayers \circdef \\%
      (((HiddenLayer(1, 3, 2)) \lpar | \lchanset layerRes11, layerRes12, layerRes13 \rchanset | \rpar HiddenLayer(2, 1, 3))  \\%
      \t1 %
      \lpar | \lchanset layerRes21 \rchanset | \rpar HiddenLayer(3,1,1)) \\%
      
      OutputLayer \circdef \\%
      (\lpar \lchanset layerRes31 \rchanset \rpar i : 1 \upto 2 \circspot Node(4, i, 1)) \\%
      
      ANN \circdef \\%
      ((HiddenLayers \lpar | \lchanset layerRes31 \rchanset | \rpar OutputLayer) \circhide ANNHiddenEvts) \circseq ANN \\%
      
      ANNRenamed \circdef \\%
      (ANN) \lcircrename layerRes01, layerRes02, layerRes41 := anewError\_in, adiff\_in, angleOutputE\_out \rcircrename \circinterrupt terminate \then \Skip \\%
      
    \circspot ANNRenamed \\%
  \circend
\end{circus}    


  \caption{$AnglePIDANN2$ Circus Program, a mock example}
  \label{anglepidann2-circus-example}
\end{figure} 

\section{CSP Semantics Sketch} 
%Figure from overleaf: 
\begin{figure}[h]
  \setlength{\zedindent}{0pt}
  \begin{zed}
    ANNRenamed = ANN \circinterrupt end \then SKIP \\
    ANN = 
    \\
    \quad
      \begin{array}[t]{l}
            ((HiddenLayers \parallel[\{| layerRes.(layerNo-1) |\}] OutputLayer) \hide ANNHiddenEvts) \\%
            \circseq ANN
          \end{array}  
    \\

    ANNHiddenEvts =  \Sigma \hide \{| layerRes.0, layerRes.layerNo, end |\}
    
    \also
    
    HiddenLayers = \Parallel i : 1\upto layerNo-1 @ [~\{| layerRes.(i-1), layerRes.i |\}~] \\
    \t2 
      HiddenLayer(i, layerSize(i), layerSize(i-1)) 
    
    \also
    
    HiddenLayer(l,s,inpSize) = 
    %\\
    %\quad
      \Parallel i : 1 \upto s @  [ \{| layerRes.(l-1) |\} ]  Node(l, i, inpSize)
    
    \also
    
    Node(l, n, inpSize) = 
    \\
    \quad 
      (\begin{array}[t]{l}    
         (\Interleave i: 1 \upto inpSize @ NodeIn(l, n, i))
         \\
         \quad
           \parallel[ \; \{| nodeOut.l.n |\} \; ] 
         \\
         Collator(l, n, inpSize)~) \hide \{| nodeOut |\}
       \end{array}  
    
    \also
    
    NodeIn(l, n, i) = layerRes.(l-1).n?x \then nodeOut.l.n.i!(x * weight) \then \Skip 
    
    \also
    
    Collator(l, n, inpSize) = \mathbf{let} \\
    \t2 C(l, n, 0, sum) = layerRes.l.n!(ReLU(sum + bias)) \then \Skip  \\
    \t2 C(l, n, i, sum) = nodeOut.l.n.i?x \then C(l, n, (i-1), (sum+x)) \\
    \t1 \mathbf{within} \\
    \t2 C(l, n, inpSize, 0) 

    \also 

    OutputLayer = \Parallel i : 1 \upto layerSize(layerNo) @  \; [ \; \{| layerRes.(layerNo-1) |\} \; ] 
    \\
    \t2
      Node(layerNo, i, layerSize(layerNo-1))   
  \end{zed}
  \caption{CSP ANN Semantic Pattern.} 
  \label{fig:annsemantics}
\end{figure}


\end{document}
