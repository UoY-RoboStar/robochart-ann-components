%Declarations, used for the real number circus theory. 
\def \real {\mathbb{R}}
\newcommand{\decimalpoint}{.}

\newcommand{\ubar}[1]{\mkern 1.5mu\underline{\mkern-1.5mu#1\mkern-1.5mu}\mkern 1.5mu}

\newcommand{\meta}[1]{\mathsf{\color{GreyColor}\ubar{#1}}}

\newcommand{\lsem}{\lbrack\!\lbrack}
\newcommand{\rsem}{\rbrack\!\rbrack}

\DeclareFontFamily{U}{mathc}{}
\DeclareFontShape{U}{mathc}{m}{it}%
{<->s*[1.03] mathc10}{}
\DeclareMathAlphabet{\mathscr}{U}{mathc}{m}{it}

\newtheorem{ruleT}{Rule}
\newcounter{tRule}%[section]
\newenvironment{TRule}[2][]{
  \begin{figure*}\centering
    \begin{minipage}{.98\textwidth}
	  \rule{\linewidth}{.1pt}
	  \refstepcounter{tRule}
	  \textbf{Rule~\thetRule.~#1\ {$#2$}}
	  \\
      \noindent\rule{\linewidth}{.1pt}
	  \footnotesize
	  \begin{displaymath}
}
{\end{displaymath}\medskip\noindent\rule{\linewidth}{.1pt}	\end{minipage}\end{figure*}}