%%%%%%%%%%%%%%%%%%%%%%%%%%%%%%%%%%%%%%%%%
% The Legrand Orange Book
% Structural Definitions File
% Version 2.0 (9/2/15)
%
% Original author:
% Mathias Legrand (legrand.mathias@gmail.com) with modifications by:
% Vel (vel@latextemplates.com)
% 
% This file has been downloaded from:
% http://www.LaTeXTemplates.com
%
% License:
% CC BY-NC-SA 3.0 (http://creativecommons.org/licenses/by-nc-sa/3.0/)
%
%%%%%%%%%%%%%%%%%%%%%%%%%%%%%%%%%%%%%%%%%

%----------------------------------------------------------------------------------------
%	VARIOUS REQUIRED PACKAGES AND CONFIGURATIONS
%----------------------------------------------------------------------------------------

\usepackage[top=3cm,bottom=3cm,left=3cm,right=3cm,headsep=10pt,a4paper]{geometry} % Page margins
%\usepackage[top=3cm,bottom=3cm,left=3.2cm,right=3.2cm,headsep=10pt,letterpaper]{geometry} % Page margins

\usepackage{graphicx} % Required for including pictures
\graphicspath{{Pictures/}} % Specifies the directory where pictures are stored

\usepackage{lipsum} % Inserts dummy text

\usepackage{tikz} % Required for drawing custom shapes

\usepackage[english]{babel} % English language/hyphenation

\usepackage[shortlabels]{enumitem} % Customize lists
\setlist{nolistsep} % Reduce spacing between bullet points and numbered lists

\usepackage{booktabs} % Required for nicer horizontal rules in tables

\usepackage{xcolor} % Required for specifying colors by name
%\definecolor{maincolour}{RGB}{243,102,25} % Define the orange color used for highlighting throughout the book
\definecolor{maincolour}{RGB}{43,102,225} % Define the blue color used for highlighting throughout the book

%----------------------------------------------------------------------------------------
%	FONTS
%----------------------------------------------------------------------------------------

\usepackage{avant} % Use the Avantgarde font for headings
%\usepackage{times} % Use the Times font for headings
\usepackage{mathptmx} % Use the Adobe Times Roman as the default text font together with math symbols from the Sym­bol, Chancery and Com­puter Modern fonts

\usepackage{microtype} % Slightly tweak font spacing for aesthetics
\usepackage[utf8]{inputenc} % Required for including letters with accents
\usepackage[T1]{fontenc} % Use 8-bit encoding that has 256 glyphs

%----------------------------------------------------------------------------------------
%	BIBLIOGRAPHY AND INDEX
%----------------------------------------------------------------------------------------

\usepackage{csquotes}
\usepackage[style=numeric,citestyle=numeric,sorting=nyt,sortcites=true,autopunct=true,autolang=hyphen,hyperref=true,abbreviate=false,backref=true,backend=biber,defernumbers=true]{biblatex}
\addbibresource{bibliography.bib} % BibTeX bibliography file
\addbibresource{espfor.bib}
%\addbibresource{robochart.bib}
\addbibresource{publications.bib} % BibTeX bibliography file
\defbibheading{bibempty}{}

\usepackage{calc} % For simpler calculation - used for spacing the index letter headings correctly
\usepackage{makeidx} % Required to make an index
\makeindex % Tells LaTeX to create the files required for indexing

%----------------------------------------------------------------------------------------
%	MAIN TABLE OF CONTENTS
%----------------------------------------------------------------------------------------

%\cftsetindents{chapter}{0em}{3em}
%\cftsetindents{section}{0em}{3em}
%\cftsetindents{subsection}{0em}{3em}
%\cftsetindents{subsubsection}{0em}{3em}

%\renewcommand{\cftpartfont}{\addvspace{20pt}\centering\large\bfseries}

%\cftpagenumbersoff{part}

\usepackage{titletoc} % Required for manipulating the table of contents
\usepackage[titles]{tocloft}
\contentsmargin{0cm} % Removes the default margin
%
% Part text styling
\titlecontents{part}[0cm]
{\addvspace{20pt}\centering\large\bfseries}
{}
{}
{}
%
% Chapter text styling
\titlecontents{chapter}[1.25cm] % Indentation
{\addvspace{12pt}\large\sffamily\bfseries} % Spacing and font options for chapters
{\color{maincolour!60}\contentslabel[\Large\thecontentslabel]{1.25cm}\color{maincolour}} % Chapter number
{\color{maincolour}}  
{\color{maincolour!60}\normalsize\;\titlerule*[.5pc]{.}\;\thecontentspage} % Page number

% Section text styling
\titlecontents{section}[1.25cm] % Indentation
{\addvspace{3pt}\sffamily\bfseries} % Spacing and font options for sections
{\contentslabel[\thecontentslabel]{1.25cm}} % Section number
{}
{\hfill\color{black}\thecontentspage} % Page number
[]

% Subsection text styling
\titlecontents{subsection}[1.25cm] % Indentation
{\addvspace{3pt}\sffamily\small} % Spacing and font options for subsections
{\contentslabel[\thecontentslabel]{1.25cm}} % Subsection number
{}
{\ \titlerule*[.5pc]{.}\;\thecontentspage} % Page number
[]

% List of figures
\titlecontents{figure}[0em]
{\addvspace{-5pt}\sffamily}
{\thecontentslabel\hspace*{1em}}
{}
{\ \titlerule*[.5pc]{.}\;\thecontentspage}
[]

% List of tables
\titlecontents{table}[0em]
{\addvspace{-5pt}\sffamily}
{\thecontentslabel\hspace*{1em}}
{}
{\ \titlerule*[.5pc]{.}\;\thecontentspage}
[]


%----------------------------------------------------------------------------------------
%	MINI TABLE OF CONTENTS IN PART HEADS
%----------------------------------------------------------------------------------------

% Chapter text styling
\titlecontents{lchapter}[0em] % Indenting
{\addvspace{15pt}\large\sffamily\bfseries} % Spacing and font options for chapters
{\color{maincolour}\contentslabel[\Large\thecontentslabel]{1.25cm}\color{maincolour}} % Chapter number
{}  
{\color{maincolour}\normalsize\sffamily\bfseries\;\titlerule*[.5pc]{.}\;\thecontentspage} % Page number

% Section text styling
\titlecontents{lsection}[0em] % Indenting
{\sffamily\small} % Spacing and font options for sections
{\contentslabel[\thecontentslabel]{1.25cm}} % Section number
{}
{}

% Subsection text styling
\titlecontents{lsubsection}[.5em] % Indentation
{\normalfont\footnotesize\sffamily} % Font settings
{}
{}
{}

%----------------------------------------------------------------------------------------
%	PAGE HEADERS
%----------------------------------------------------------------------------------------

\usepackage{fancyhdr} % Required for header and footer configuration

\pagestyle{fancy}
\renewcommand{\chaptermark}[1]{\markboth{\sffamily\normalsize\bfseries\chaptername\ \thechapter.\ #1}{}} % Chapter text font settings
\renewcommand{\sectionmark}[1]{\markright{\sffamily\normalsize\thesection\hspace{5pt}#1}{}} % Section text font settings
\fancyhf{} \fancyhead[LE,RO]{\sffamily\normalsize\thepage} % Font setting for the page number in the header
\fancyhead[LO]{\rightmark} % Print the nearest section name on the left side of odd pages
\fancyhead[RE]{\leftmark} % Print the current chapter name on the right side of even pages
\renewcommand{\headrulewidth}{0.5pt} % Width of the rule under the header
\addtolength{\headheight}{2.5pt} % Increase the spacing around the header slightly
\renewcommand{\footrulewidth}{0pt} % Removes the rule in the footer
\fancypagestyle{plain}{\fancyhead{}\renewcommand{\headrulewidth}{0pt}} % Style for when a plain pagestyle is specified

% Removes the header from odd empty pages at the end of chapters
\makeatletter
\renewcommand{\cleardoublepage}{
\clearpage\ifodd\c@page\else
\hbox{}
\vspace*{\fill}
\thispagestyle{empty}
\newpage
\fi}

%----------------------------------------------------------------------------------------
%	THEOREM STYLES
%----------------------------------------------------------------------------------------

\usepackage{amsmath,amsfonts,amssymb,amsthm} % For math equations, theorems, symbols, etc

\newcommand{\intoo}[2]{\mathopen{]}#1\,;#2\mathclose{[}}
\newcommand{\ud}{\mathop{\mathrm{{}d}}\mathopen{}}
\newcommand{\intff}[2]{\mathopen{[}#1\,;#2\mathclose{]}}
\newtheorem{notation}{Notation}[chapter]

% Boxed/framed environments

\newtheoremstyle{rulebox}% % Theorem style name
{0pt}% Space above
{0pt}% Space below
{\normalfont}% % Body font
{}% Indent amount
{\small\bf\sffamily\color{maincolour}}% % Theorem head font
{\;}% Punctuation after theorem head
{0.25em}% Space after theorem head
%{\small\sffamily\color{maincolour}#3\vspace{0.5cm}\\}
{\small\sffamily\color{maincolour}\thmname{#1}\nobreakspace\thmnumber{\@ifnotempty{#1}{}\@upn{#2}}.% Theorem text (e.g. Theorem 2.1)
	\thmnote{\nobreakspace\the\thm@notefont\sffamily\bfseries\color{black}\nobreakspace#3\\}}

\newtheoremstyle{syntaxbox}% % Theorem style name
{0pt}% Space above
{0pt}% Space below
{\normalfont}% % Body font
{}% Indent amount
{\small\bf\sffamily\color{maincolour}}% % Theorem head font
{\;}% Punctuation after theorem head
{0.25em}% Space after theorem head
%{\small\sffamily\color{maincolour}#3\vspace{0.5cm}\\}
{\small\sffamily\color{maincolour}\thmname{#1}\nobreakspace\thmnumber{\@ifnotempty{#1}{}\@upn{#2}}% Theorem text (e.g. Theorem 2.1)
\thmnote{\nobreakspace\the\thm@notefont\sffamily\bfseries\color{black}---\nobreakspace#3.\\}}

%\newtheoremstyle{tablebox}% % Theorem style name
%{0pt}% Space above
%{0pt}% Space below
%{\normalfont}% % Body font
%{}% Indent amount
%{\small\bf\sffamily\color{maincolour}}% % Theorem head font
%{\;}% Punctuation after theorem head
%{0.25em}% Space after theorem head
%%{\small\sffamily\color{maincolour}#3\vspace{0.5cm}\\}
%{\small\sffamily\color{maincolour}\thmname{#1}\nobreakspace\thmnumber{\@ifnotempty{#1}{}\@upn{#2}}% Theorem text (e.g. Theorem 2.1)
%	\thmnote{\nobreakspace\the\thm@notefont\sffamily\bfseries\color{black}---\nobreakspace#3.\\}}

\newtheoremstyle{maincolournumbox}% % Theorem style name
{0pt}% Space above
{0pt}% Space below
{\normalfont}% % Body font
{}% Indent amount
{\small\bf\sffamily\color{maincolour}}% % Theorem head font
{\;}% Punctuation after theorem head
{0.25em}% Space after theorem head
{\small\sffamily\color{maincolour}\thmname{#1}\nobreakspace\thmnumber{\@ifnotempty{#1}{}\@upn{#2}}% Theorem text (e.g. Theorem 2.1)
\thmnote{\nobreakspace\the\thm@notefont\sffamily\bfseries\color{black}---\nobreakspace#3.}} % Optional theorem note
\renewcommand{\qedsymbol}{$\blacksquare$}% Optional qed square

\newtheoremstyle{blacknumex}% Theorem style name
{5pt}% Space above
{5pt}% Space below
{\normalfont}% Body font
{} % Indent amount
{\small\bf\sffamily}% Theorem head font
{\;}% Punctuation after theorem head
{0.25em}% Space after theorem head
{\small\sffamily{\tiny\ensuremath{\blacksquare}}\nobreakspace\thmname{#1}\nobreakspace\thmnumber{\@ifnotempty{#1}{}\@upn{#2}}% Theorem text (e.g. Theorem 2.1)
\thmnote{\nobreakspace\the\thm@notefont\sffamily\bfseries---\nobreakspace#3.}}% Optional theorem note

\newtheoremstyle{blacknumbox} % Theorem style name
{0pt}% Space above
{0pt}% Space below
{\normalfont}% Body font
{}% Indent amount
{\small\bf\sffamily}% Theorem head font
{\;}% Punctuation after theorem head
{0.25em}% Space after theorem head
{\small\sffamily\thmname{#1}\nobreakspace\thmnumber{\@ifnotempty{#1}{}\@upn{#2}}% Theorem text (e.g. Theorem 2.1)
\thmnote{\nobreakspace\the\thm@notefont\sffamily\bfseries---\nobreakspace#3.}}% Optional theorem note

% Non-boxed/non-framed environments
\newtheoremstyle{maincolournum}% % Theorem style name
{5pt}% Space above
{5pt}% Space below
{\normalfont}% % Body font
{}% Indent amount
{\small\bf\sffamily\color{maincolour}}% % Theorem head font
{\;}% Punctuation after theorem head
{0.25em}% Space after theorem head
{\small\sffamily\color{maincolour}\thmname{#1}\nobreakspace\thmnumber{\@ifnotempty{#1}{}\@upn{#2}}% Theorem text (e.g. Theorem 2.1)
\thmnote{\nobreakspace\the\thm@notefont\sffamily\bfseries\color{black}---\nobreakspace#3.}} % Optional theorem note
\renewcommand{\qedsymbol}{$\blacksquare$}% Optional qed square
\makeatother

% Defines the theorem text style for each type of theorem to one of the three styles above
\newcounter{dummy} 
\numberwithin{dummy}{section}

\newcounter{syntax}
\numberwithin{syntax}{section}

\theoremstyle{syntaxbox}
\newtheorem{syntaxT}{Syntax}
\newtheorem{syntaxTT}[syntax]{Syntax}
\newtheorem*{syntaxT*}{Syntax}

\theoremstyle{rulebox}
\newtheorem{ruleT}{Rule}
\newtheorem*{ruleT*}{Rule}


%\theoremstyle{tablebox}
%\newtheorem{tableT}{Table}[chapter]
%\newtheorem*{tableT*}{Table}


\theoremstyle{maincolournumbox}
\newtheorem{theoremeT}[dummy]{Theorem}
\newtheorem{problem}{Problem}[chapter]
\newtheorem{exercise}{Exercise}[chapter]
\newtheorem{exerciseT}{Exercise}[chapter]
\theoremstyle{blacknumex}
\newtheorem{exampleT}{Example}[chapter]
\theoremstyle{blacknumbox}
\newtheorem{vocabulary}{Vocabulary}[chapter]
\newtheorem{definitionT}{Definition}[section]
\newtheorem{corollaryT}[dummy]{Corollary}
\theoremstyle{maincolournum}
\newtheorem{proposition}[dummy]{Proposition}

%----------------------------------------------------------------------------------------
%	DEFINITION OF COLORED BOXES
%----------------------------------------------------------------------------------------

\RequirePackage[framemethod=default]{mdframed} % Required for creating the theorem, definition, exercise and corollary boxes

\newlength{\mytopskip}
\setlength{\mytopskip}{\topskip+5pt}

% table
\newmdenv[skipabove=\baselineskip,
skipbelow=7pt,
backgroundcolor=black!5,
linecolor=maincolour,linewidth=0.8pt,
innerleftmargin=5pt,
innerrightmargin=5pt,
innertopmargin=5pt,
%innertopmargin=\mytopskip,
leftmargin=0cm,
rightmargin=0cm,
nobreak=true,
innerbottommargin=5pt]{tblBox}

% Rule box
\newmdenv[skipabove=\baselineskip,
skipbelow=7pt,
backgroundcolor=black!5,
linecolor=maincolour,linewidth=0.8pt,
innerleftmargin=5pt,
innerrightmargin=5pt,
%innertopmargin=5pt,
innertopmargin=\mytopskip,
leftmargin=0cm,
rightmargin=0cm,
nobreak=true,
innerbottommargin=5pt]{rBox}

% Syntax box
\newmdenv[skipabove=\baselineskip,
skipbelow=7pt,
backgroundcolor=black!5,
linecolor=maincolour,linewidth=0.8pt,
innerleftmargin=5pt,
innerrightmargin=5pt,
%innertopmargin=5pt,
innertopmargin=\mytopskip,
leftmargin=0cm,
rightmargin=0cm,
nobreak=true,
innerbottommargin=5pt]{sBox}

% Theorem box
\newmdenv[skipabove=7pt,
skipbelow=7pt,
backgroundcolor=black!5,
linecolor=maincolour,linewidth=0.8pt,
innerleftmargin=5pt,
innerrightmargin=5pt,
innertopmargin=5pt,
leftmargin=0cm,
rightmargin=0cm,
innerbottommargin=5pt]{tBox}

% Exercise box	  
\newmdenv[skipabove=7pt,
skipbelow=7pt,
rightline=false,
leftline=true,
topline=false,
bottomline=false,
backgroundcolor=maincolour!10,
linecolor=maincolour,
innerleftmargin=5pt,
innerrightmargin=5pt,
innertopmargin=5pt,
innerbottommargin=5pt,
leftmargin=0cm,
rightmargin=0cm,
linewidth=4pt]{eBox}

% Definition box
\newmdenv[skipabove=7pt,
skipbelow=0pt,
rightline=false,
leftline=true,
topline=false,
bottomline=false,
linecolor=maincolour,
innerleftmargin=5pt,
innerrightmargin=5pt,
innertopmargin=15pt,
leftmargin=0cm,
rightmargin=0cm,
linewidth=4pt,
innerbottommargin=0pt]{dBox}	

% Corollary box
\newmdenv[skipabove=7pt,
skipbelow=7pt,
rightline=false,
leftline=true,
topline=false,
bottomline=false,
linecolor=gray,
backgroundcolor=black!5,
innerleftmargin=5pt,
innerrightmargin=5pt,
innertopmargin=5pt,
leftmargin=0cm,
rightmargin=0cm,
linewidth=4pt,
innerbottommargin=5pt]{cBox}

% Example box
\newmdenv[skipabove=\baselineskip,
skipbelow=7pt,
backgroundcolor=black!5,
linecolor=maincolour,linewidth=0.8pt,
innerleftmargin=5pt,
innerrightmargin=5pt,
%innertopmargin=5pt,
innertopmargin=\mytopskip,
leftmargin=0cm,
rightmargin=0cm,
nobreak=true,
innerbottommargin=5pt]{xBox}

% Creates an environment for each type of theorem and assigns it a theorem text style from the "Theorem Styles" section above and a colored box from above

\newcounter{tblcnt}[chapter]

\newenvironment{syntax}{\begin{sBox}\begin{syntaxT}}{\end{syntaxT}\end{sBox}}
\newenvironment{psyntax}{\begin{sBox}\begin{syntaxTT}}{\end{syntaxTT}\end{sBox}}
\newenvironment{syntax*}{\begin{sBox}\begin{syntaxT*}}{\end{syntaxT*}\end{sBox}}

\newenvironment{TRuleAux}{\begin{rBox}\begin{ruleT}}{\end{ruleT}\end{rBox}}
\newenvironment{TRuleAux*}{\begin{rBox}\begin{ruleT*}}{\end{ruleT*}\end{rBox}}

\renewcommand{\thetblcnt}{\arabic{chapter}.\arabic{tblcnt}}
\newenvironment{tablebox}[1]{
	\begin{tblBox}
		\refstepcounter{tblcnt}
		\newcommand{\tbltitle}{
			{\footnotesize\sffamily\color{maincolour}Table\nobreakspace \arabic{chapter}.\arabic{tblcnt}% Theorem text (e.g. Theorem 2.1)
			\nobreakspace\sffamily\bfseries: \nobreakspace#1.}
		}
}
{	\\
	\centering\tbltitle
	\end{tblBox}
}
\newcommand{\tblheader}[1]{\textbf{\textcolor{maincolour}{#1}}}
%\newenvironment{tablebox*}{\begin{tblBox}\begin{tableT*}}{\end{tableT*}\end{tblBox}}

\newenvironment{theorem}{\begin{tBox}\begin{theoremeT}}{\end{theoremeT}\end{tBox}}
\newenvironment{exerciseBox}{\begin{eBox}\begin{exerciseT}}{\hfill{\color{maincolour}\tiny\ensuremath{\blacksquare}}\end{exerciseT}\end{eBox}}				  
\newenvironment{definition}{\begin{dBox}\begin{definitionT}}{\end{definitionT}\end{dBox}}	
%\newenvironment{example}{\begin{exampleT}}{\hfill{\tiny\ensuremath{\blacksquare}}\end{exampleT}}
\newenvironment{example}{\begin{xBox}\begin{exampleT}}{\hfill{\tiny\ensuremath{\blacksquare}}\end{exampleT}\end{xBox}}
\newenvironment{corollary}{\begin{cBox}\begin{corollaryT}}{\end{corollaryT}\end{cBox}}	

%----------------------------------------------------------------------------------------
%	REMARK ENVIRONMENT
%----------------------------------------------------------------------------------------

\newenvironment{remark}{\par\vspace{10pt}\small % Vertical white space above the remark and smaller font size
\begin{list}{}{
\leftmargin=35pt % Indentation on the left
\rightmargin=25pt}\item\ignorespaces % Indentation on the right
\makebox[-2.5pt]{\begin{tikzpicture}[overlay]
\node[draw=maincolour!60,line width=1pt,circle,fill=maincolour!25,font=\sffamily\bfseries,inner sep=2pt,outer sep=0pt] at (-15pt,0pt){\textcolor{maincolour}{R}};\end{tikzpicture}} % Orange R in a circle
\advance\baselineskip -1pt}{\end{list}\vskip5pt} % Tighter line spacing and white space after remark

%----------------------------------------------------------------------------------------
%	SECTION NUMBERING IN THE MARGIN
%----------------------------------------------------------------------------------------

\makeatletter
\renewcommand{\@seccntformat}[1]{\llap{\textcolor{maincolour}{\csname the#1\endcsname}\hspace{1em}}}                    
\renewcommand{\section}{\@startsection{section}{1}{\z@}
{-4ex \@plus -1ex \@minus -.4ex}
{1ex \@plus.2ex }
{\normalfont\large\sffamily\bfseries}}
\renewcommand{\subsection}{\@startsection {subsection}{2}{\z@}
{-3ex \@plus -0.1ex \@minus -.4ex}
{0.5ex \@plus.2ex }
{\normalfont\sffamily\bfseries}}
\renewcommand{\subsubsection}{\@startsection {subsubsection}{3}{\z@}
{-2ex \@plus -0.1ex \@minus -.2ex}
{.2ex \@plus.2ex }
{\normalfont\small\sffamily\bfseries}}                        
\renewcommand\paragraph{\@startsection{paragraph}{4}{\z@}
{-2ex \@plus-.2ex \@minus .2ex}
{.1ex}
{\normalfont\small\sffamily\bfseries}}

%----------------------------------------------------------------------------------------
%	PART HEADINGS
%----------------------------------------------------------------------------------------

% numbered part in the table of contents
\newcommand{\@mypartnumtocformat}[2]{%
\setlength\fboxsep{0pt}%
\noindent\colorbox{maincolour!20}{\strut\parbox[c][.7cm]{\ecart}{\color{maincolour!70}\Large\sffamily\bfseries\centering#1}}\hskip\esp\colorbox{maincolour!40}{\strut\parbox[c][.7cm]{\linewidth-\ecart-\esp}{\Large\sffamily\centering#2}}}%
%%%%%%%%%%%%%%%%%%%%%%%%%%%%%%%%%%
% unnumbered part in the table of contents
\newcommand{\@myparttocformat}[1]{%
\setlength\fboxsep{0pt}%
\noindent\colorbox{maincolour!40}{\strut\parbox[c][.7cm]{\linewidth}{\Large\sffamily\centering#1}}}%
%%%%%%%%%%%%%%%%%%%%%%%%%%%%%%%%%%
\newlength\esp
\setlength\esp{4pt}
\newlength\ecart
\setlength\ecart{1.2cm-\esp}
\newcommand{\thepartimage}{}%
\newcommand{\partimage}[1]{\renewcommand{\thepartimage}{#1}}%
\def\@part[#1]#2{%
\ifnum \c@secnumdepth >-2\relax%
\refstepcounter{part}%
\addcontentsline{toc}{part}{\texorpdfstring{\protect\@mypartnumtocformat{\thepart}{#1}}{\partname~\thepart\ ---\ #1}}
\else%
\addcontentsline{toc}{part}{\texorpdfstring{\protect\@myparttocformat{#1}}{#1}}%
\fi%
\startcontents%
\markboth{}{}%
{\thispagestyle{empty}%
\begin{tikzpicture}[remember picture,overlay]%
\node at (current page.north west){\begin{tikzpicture}[remember picture,overlay]%	
\fill[maincolour!20](0cm,0cm) rectangle (\paperwidth,-\paperheight);
\node[anchor=north] at (4cm,-3.25cm){\color{maincolour!40}\fontsize{220}{100}\sffamily\bfseries\@Roman\c@part}; 
\node[anchor=south east] at (\paperwidth-1cm,-\paperheight+1cm){\parbox[t][][t]{8.5cm}{
\printcontents{l}{0}{\setcounter{tocdepth}{1}}%
}};
\node[anchor=north east] at (\paperwidth-1.5cm,-3.25cm){\parbox[t][][t]{15cm}{\strut\raggedleft\color{white}\fontsize{30}{30}\sffamily\bfseries#2}};
\end{tikzpicture}};
\end{tikzpicture}}%
\@endpart}
\def\@spart#1{%
\startcontents%
\phantomsection
{\thispagestyle{empty}%
\begin{tikzpicture}[remember picture,overlay]%
\node at (current page.north west){\begin{tikzpicture}[remember picture,overlay]%	
\fill[maincolour!20](0cm,0cm) rectangle (\paperwidth,-\paperheight);
\node[anchor=north east] at (\paperwidth-1.5cm,-3.25cm){\parbox[t][][t]{15cm}{\strut\raggedleft\color{white}\fontsize{30}{30}\sffamily\bfseries#1}};
\end{tikzpicture}};
\end{tikzpicture}}
\addcontentsline{toc}{part}{\texorpdfstring{%
\setlength\fboxsep{0pt}%
\noindent\protect\colorbox{maincolour!40}{\strut\protect\parbox[c][.7cm]{\linewidth}{\Large\sffamily\protect\centering #1\quad\mbox{}}}}{#1}}%
\@endpart}
\def\@endpart{\vfil\newpage
\if@twoside
\if@openright
\null
\thispagestyle{empty}%
\newpage
\fi
\fi
\if@tempswa
\twocolumn
\fi}

%----------------------------------------------------------------------------------------
%	CHAPTER HEADINGS - IMAGES
%----------------------------------------------------------------------------------------

% A switch to conditionally include a picture, implemented by  Christian Hupfer
\newif\ifusechaptercolour
\usechaptercolourtrue
\newcommand{\thechaptercolour}{}%
\newcommand{\chaptercolour}[1]{\ifusechaptercolour\renewcommand{\thechaptercolour}{#1}\fi}%
\def\@makechapterhead#1{%
{\parindent \z@ \raggedright \normalfont
\ifnum \c@secnumdepth >\m@ne
\if@mainmatter
\begin{tikzpicture}[remember picture,overlay]
\node at (current page.north west)
{\begin{tikzpicture}[remember picture,overlay]
\node[anchor=north west,inner sep=0pt] at (0,0) {
%	\ifusechaptercolour\includegraphics[width=\paperwidth]{\thechaptercolour}\fi
	\begin{tikzpicture}
		\fill[\thechaptercolour] (0,0) rectangle (\paperwidth,0.5\paperwidth);
	\end{tikzpicture}
};
\draw[anchor=west] (\Gm@lmargin,-9cm) node [line width=2pt,rounded corners=15pt,draw=maincolour,fill=white,fill opacity=0.5,inner sep=15pt]{\strut\makebox[22cm]{}};
\draw[anchor=west] (\Gm@lmargin+.3cm,-9.1cm) node {\huge\sffamily\bfseries\color{black}\thechapter. #1\strut};
\end{tikzpicture}};
\end{tikzpicture}
\else
\begin{tikzpicture}[remember picture,overlay]
\node at (current page.north west)
{\begin{tikzpicture}[remember picture,overlay]
\node[anchor=north west,inner sep=0pt] at (0,0) {
	%\ifusechaptercolour\includegraphics[width=\paperwidth]{\thechaptercolour}\fi
	\begin{tikzpicture}
		\fill[\thechaptercolour] (0,0) rectangle (\paperwidth,0.5\paperwidth);
	\end{tikzpicture}

};
\draw[anchor=west] (\Gm@lmargin,-9cm) node [line width=2pt,rounded corners=15pt,draw=maincolour,fill=white,fill opacity=0.5,inner sep=15pt]{\strut\makebox[22cm]{}};
\draw[anchor=west] (\Gm@lmargin+.3cm,-9.1cm) node {\huge\sffamily\bfseries\color{black}#1\strut};
\end{tikzpicture}};
\end{tikzpicture}
\fi\fi\par\vspace*{270\p@}}}

%-------------------------------------------

\def\@makeschapterhead#1{%
\begin{tikzpicture}[remember picture,overlay]
\node at (current page.north west)
{\begin{tikzpicture}[remember picture,overlay]
\node[anchor=north west,inner sep=0pt] at (0,0) {
%	\ifusechaptercolour\includegraphics[width=\paperwidth]{\thechaptercolour}\fi
	\begin{tikzpicture}
		\fill[\thechaptercolour] (0,0) rectangle (\paperwidth,0.5\paperwidth);
	\end{tikzpicture}

};
\draw[anchor=west] (\Gm@lmargin,-9cm) node [line width=2pt,rounded corners=15pt,draw=maincolour,fill=white,fill opacity=0.5,inner sep=15pt]{\strut\makebox[22cm]{}};
\draw[anchor=west] (\Gm@lmargin+.3cm,-9cm) node {\huge\sffamily\bfseries\color{black}#1\strut};
\end{tikzpicture}};
\end{tikzpicture}
\par\vspace*{270\p@}}
\makeatother

%%----------------------------------------------------------------------------------------
%%	CHAPTER HEADINGS - COLOUR
%%----------------------------------------------------------------------------------------
%
%% A switch to conditionally include a picture, implemented by  Christian Hupfer
%\newif\ifusechaptercolour
%\usechaptercolourtrue
%\newcommand{\thechaptercolour}{}%
%\newcommand{\chaptercolour}[1]{\ifusechaptercolour\renewcommand{\thechaptercolour}{#1}\fi}%
%\def\@makechapterhead#1{%
%	{\parindent \z@ \raggedright \normalfont
%		\ifnum \c@secnumdepth >\m@ne
%		\if@mainmatter
%		\begin{tikzpicture}[remember picture,overlay]
%		\node at (current page.north west)
%		{\begin{tikzpicture}[remember picture,overlay]
%			\node[anchor=north west,inner sep=0pt] at (0,0) {
%				%\ifusechaptercolour\includegraphics[width=\paperwidth]{\thechaptercolour}\fi
%				\begin{tikzpicture}
%				\fill[blue!40!white] (0,0) rectangle (\paperwidth,0.5\paperwidth);
%				\end{tikzpicture}
%			};
%			\draw[anchor=west] (\Gm@lmargin,-9cm) node [line width=2pt,rounded corners=15pt,draw=maincolour,fill=white,fill opacity=0.5,inner sep=15pt]{\strut\makebox[22cm]{}};
%			\draw[anchor=west] (\Gm@lmargin+.3cm,-9cm) node {\huge\sffamily\bfseries\color{black}\thechapter. #1\strut};
%	\end{tikzpicture}};
%\end{tikzpicture}
%\else
%\begin{tikzpicture}[remember picture,overlay]
%\node at (current page.north west)
%{\begin{tikzpicture}[remember picture,overlay]
%	\node[anchor=north west,inner sep=0pt] at (0,0) {\ifusechaptercolour\includegraphics[width=\paperwidth]{\thechaptercolour}\fi};
%	\draw[anchor=west] (\Gm@lmargin,-9cm) node [line width=2pt,rounded corners=15pt,draw=maincolour,fill=white,fill opacity=0.5,inner sep=15pt]{\strut\makebox[22cm]{}};
%	\draw[anchor=west] (\Gm@lmargin+.3cm,-9cm) node {\huge\sffamily\bfseries\color{black}#1\strut};
%	\end{tikzpicture}};
%\end{tikzpicture}
%\fi\fi\par\vspace*{270\p@}}}
%
%%-------------------------------------------
%
%\def\@makeschapterhead#1{%
%\begin{tikzpicture}[remember picture,overlay]
%\node at (current page.north west)
%{\begin{tikzpicture}[remember picture,overlay]
%\node[anchor=north west,inner sep=0pt] at (0,0) {\ifusechaptercolour\includegraphics[width=\paperwidth]{\thechaptercolour}\fi};
%\draw[anchor=west] (\Gm@lmargin,-9cm) node [line width=2pt,rounded corners=15pt,draw=maincolour,fill=white,fill opacity=0.5,inner sep=15pt]{\strut\makebox[22cm]{}};
%\draw[anchor=west] (\Gm@lmargin+.3cm,-9cm) node {\huge\sffamily\bfseries\color{black}#1\strut};
%\end{tikzpicture}};
%\end{tikzpicture}
%\par\vspace*{270\p@}}
%\makeatother

%----------------------------------------------------------------------------------------
%	HYPERLINKS IN THE DOCUMENTS
%----------------------------------------------------------------------------------------

\usepackage{hyperref}
\hypersetup{hidelinks,colorlinks=false,breaklinks=true,urlcolor= maincolour,bookmarksopen=false,pdftitle={Title},pdfauthor={Author}}
\usepackage{bookmark}
\bookmarksetup{
open,
numbered,
addtohook={%
\ifnum\bookmarkget{level}=0 % chapter
\bookmarksetup{bold}%
\fi
\ifnum\bookmarkget{level}=-1 % part
\bookmarksetup{color=maincolour,bold}%
\fi
}
}


%----------------------------------------------------------------------------------------
% TABLES
%----------------------------------------------------------------------------------------

%\usepackage{shortvrb}
%\MakeShortVerb{\|}
\usepackage{tabularx}
\usepackage{longtable}
\setlist{nolistsep}
\usepackage{colortbl}
\newcommand\Tstrut{\rule{0pt}{3ex}}         % = `top' strut
%\newcolumntype{|}{!{\color{maincolour}\vrule width 0.8pt}}
\arrayrulecolor{maincolour}
%\renewcommand{\familydefault}{\sfdefault}
%\renewcommand{\arraystretch}{1.5}

%----------------------------------------------------------------------------------------
% RULES
%----------------------------------------------------------------------------------------

\newcommand{\listrulesname}{List of rules}
\newlistof{rules}{env}{\listrulesname}

%% Chapter text styling
%\titlecontents{envsection}[0em] % Indenting
%{\addvspace{15pt}\large\sffamily\bfseries} % Spacing and font options for chapters
%{\color{maincolour}\contentslabel[\Large\thecontentslabel]{1.25cm}\color{red}} % Chapter number
%{}  
%{\color{red}\normalsize\sffamily\bfseries\;\titlerule*[.5pc]{.}\;\thecontentspage} % Page number

\newenvironment{TRule}[3][]{
	\begin{TRuleAux}[#1]\footnotesize
		\refstepcounter{tRule}
		\linkdest{rule:\thechapter:\thetRule}
		\index[rules]{#3|linktorule{rule:\thechapter:\thetRule}}
		\addcontentsline{env}{subsection}{\protect\numberline{\thetRule}~#1}% 
		%\noindent\textbf{Rule~\thetRule.~#1\vspace{-.5cm}
		\vspace{-0.8cm}%
			{\setlength{\zedindent}{0pt}\[#2\]}\vspace{-1cm}\\
%		\specialrule{0.8pt}{0em}{0em}\vspace{-0.3cm}%
		\vspace{-0.4cm}%
		\setlength{\zedindent}{0pt}%\footnotesize
		\begin{displaymath}
		%{\setlength{\zedindent}{0pt}\footnotesize
		%	\begin{displaymath}
	}{\end{displaymath}\end{TRuleAux}}
%\newenvironment{TRule*}{\begin{tRule*}\footnotesize\vspace{.2cm}}{\vspace{.2cm}\end{tRule*}}

%----------------------------------------------------------------------------------------
% PREAMBLE TO MAIN FILE
%----------------------------------------------------------------------------------------

\usepackage{eso-pic}
%\usepackage{exercise}
\usepackage{exsheets}
\usepackage[export]{adjustbox}

\DeclareInstance{exsheets-heading}{mylist}{default}{
	runin = true ,
	points-pre-code = ( ,
	points-post-code = )\space ,
	join =
	{
		main[r,vc]title[r,vc](0pt,0pt) ;
		main[r,vc]number[l,vc](.333em,0pt) ;
		main[r,vc]points[l,vc](1em,0pt)
	}
}

%\DeclareInstance{exsheets-heading}{mylist}{default}{
%	runin = true ,
%	attach = {
%		main[l,vc]number[r,vc](-0.6em,0pt) ;
%		main[r,vc]points[l,vc](\linewidth+\marginparsep,0pt)
%	}
%}

%\DeclareInstance{exsheets-heading}{mylist}{default}{
%	runin = true ,
%	attach = {
%		main[l,vc]number[l,vc](-3em,0pt) ; % 3em = indent of question body
%		main[r,vc]points[l,vc](\linewidth+\marginparsep,0pt)
%	}
%}

\SetupExSheets{
	headings = mylist , % use the new headings instance
	%	headings-format = \normalfont ,
	counter-format = ch.qu. ,
	counter-within = chapter
}

\usepackage{setspace}

%%%%%%%%%%%%%%%%%%%%%%%%%%%%%%%%%%%%%%%%%%%%%%%%%%%%%%%%%%%%%%%%%%%

\newcommand{\keywords}[1]{\par\addvspace\baselineskip
	\noindent\keywordname\enspace\ignorespaces#1}

%% NOTE (Pedro): Added this manually, even though I suppose the circus package
%% 				 should be used instead?

\makeatletter
\DeclareOldFontCommand{\rm}{\normalfont\rmfamily}{\mathrm}
\DeclareOldFontCommand{\sf}{\normalfont\sffamily}{\mathsf}
\DeclareOldFontCommand{\tt}{\normalfont\ttfamily}{\mathtt}
\DeclareOldFontCommand{\bf}{\normalfont\bfseries}{\mathbf}
\DeclareOldFontCommand{\it}{\normalfont\itshape}{\mathit}
\DeclareOldFontCommand{\sl}{\normalfont\slshape}{\@nomath\sl}
\DeclareOldFontCommand{\sc}{\normalfont\scshape}{\@nomath\sc}
\makeatother

%%%%%%%%%%%%%%%%%%%%%%%%%%%%%%%%%%%%%%%%%%%%%%%%%%%%%%%%%%%%%%%%%%%%%
%%%%%%%%%%%%%%%%%%%%%%%%%%%%%%%%%%%%%%%%%%%%%%%%%%%%%%%%%%%%%%%%%%%%%

%%%%%%%%%%%%%%%%%%%%%%%%%%%%%%%%%%%%%%%%%%%%%%%%%%%%%%%%%%%%%%%%%%%%%
%%%%%%%%%%%%%%%%%%%%%%%%%%%%%%%%%%%%%%%%%%%%%%%%%%%%%%%%%%%%%%%%%%%%%

%\widowpenalty=100000000
%\clubpenalty=1000000

\newcommand{\robochart}{{\sf \it RoboChart}}
\newcommand{\Circus}{{\sf\slshape Circus}}
\newcommand{\CircusTime}{{\sf\slshape Circus Time}}
\newcommand{\RC}[1]{{\sf #1}}


\onehalfspacing
\setlength{\parindent}{0em}
\setlength{\parskip}{1em}

%\setlength{\parindent}{0em}
%\setlength{\parskip}{1em}

%%%%%%%%%%%%%%%%%%%%%%%%%%%%%%%%%%%%%%%%%%%%%%%%%%%%%%%%%%%%%%%%%%%%%
%%%%%%%%%%%%%%%%%%%%%%%%%%%%%%%%%%%%%%%%%%%%%%%%%%%%%%%%%%%%%%%%%%%%%

\usepackage{listings}

\lstdefinestyle{ocl}{
	belowcaptionskip=1\baselineskip,
	breaklines=true,
	tabsize=2,
	frame=N,
	xleftmargin=\parindent,
	language=ocl,
	showstringspaces=false,
	basicstyle=\footnotesize\ttfamily,
	keywordstyle=\bfseries\color{black},%%\color{green!40!black},
	commentstyle=\itshape\color{gray!40!black},
	morecomment=[f][\itshape\color{gray}][0]{//},
	identifierstyle=\color{blue},
	stringstyle=\color{orange},
	morekeywords={attribute,property,package,class,extends,import,abstract, ordered, composes, unique, enum, literal, serializable}, 
}

\lstset{style=ocl}

%%%%%%%%%%%%%%%%%%%%%%%%%%%%%%%%%%%%%%%%%%%%%%%%%%%%%%%%%%%%%%%%%%%%%
%%%%%%%%%%%%%%%%%%%%%%%%%%%%%%%%%%%%%%%%%%%%%%%%%%%%%%%%%%%%%%%%%%%%%

%%%%%%%%%%%%%%%%%%%%%%%%%%%%%%%%%%%%%%%%%%%%%%%%%%%%%%%%%%%%%%%%%%%%%
%%%%%%%%%%%%%%%%%%%%%%%%%%%%%%%%%%%%%%%%%%%%%%%%%%%%%%%%%%%%%%%%%%%%%
\usepackage[zed,color]{csp}
\usepackage{imakeidx}

\onehalfspacing

\newcommand{\lrename}{~[\![~}
\newcommand{\rrename}{~]\!]~}

%%%%% commands to add to index with links
\newcommand{\BH}[3]{\large\textbf{\hyperlink{#1:#2}{#3\textsuperscript{#2}}}\normalsize}
%\newcommand{\IN}[1]{\index{#1|BH}}

\newcounter{test:test}
\DeclareDocumentCommand{\foocmd}{ o o m }{
	\stepcounter{test:test}
	\arabic{test:test}
	\IfNoValueF{#1}{#1~}\IfNoValueF{#2}{#2~}#3}

% arguments: (1) comment, (2) rule name and (3) subtype
\DeclareDocumentCommand{\createrule}{ o m o}{
	\newcounter{\IfNoValueF{#3}{#3:}#2}
	\expandafter\def\csname #2\IfValueT{#3}{#3}\endcsname{\stepcounter{\IfNoValueF{#3}{#3:}#2}
		\hyperref[semantics:#2\IfValueT{#3}{:#3}]{#2}^{\arabic{\IfNoValueF{#3}{#3:}#2}
			\linkdest{\IfNoValueF{#3}{#3:}#2:\arabic{\IfNoValueF{#3}{#3:}#2}}}
		\index[usages]{#2
			\IfValueTF{#1}{
				\IfValueTF{#3}{(#1,#3)}{(#1)}
			}{
				\IfValueT{#3}{(#3)}
			}|BH{\IfNoValueF{#3}{#3:}#2}{\arabic{\IfNoValueF{#3}{#3:}#2}}}}
}

% arguments: (1) comment, (2) rule name, (3) math symbol, and (4) subtype
\DeclareDocumentCommand{\createrulealt}{ o m m o}{
	\newcounter{\IfNoValueF{#4}{#4:}#2}
	\expandafter\def\csname #2\IfValueT{#4}{#4}\endcsname{\stepcounter{\IfNoValueF{#4}{#4:}#2}
		\hyperref[semantics:#2\IfValueT{#4}{:#4}]{#3}^{\arabic{\IfNoValueF{#4}{#4:}#2}
			\linkdest{\IfNoValueF{#4}{#4:}#2:\arabic{\IfNoValueF{#4}{#4:}#2}}}
		\index[usages]{#3
			\IfValueTF{#1}{
				\IfValueTF{#4}{(#1,#4)}{(#1)}
			}{
				\IfValueT{#4}{(#4)}
			}|BH{\IfNoValueF{#4}{#4:}#2}{\arabic{\IfNoValueF{#4}{#4:}#2}}}}
}

%\newcommand{\createrule}[2][]{
%	\newcounter{#2}
%	\expandafter\def\csname #2\endcsname{\stepcounter{#2}
%		\hyperref[semantics:#2]{#2}^{\arabic{#2}\linkdest{#2:\arabic{#2}}}\index[usages]{#2\ifx\relax#1\relax\else~(#1)\fi|BH{#2}{\arabic{#2}}}}
%}
\newenvironment{metalet}{\meta{let}\\\quad\begin{array}{l}}{\end{array}}
\newenvironment{metain}{\meta{in}\\\quad\begin{array}{l}}{\end{array}}
\newcommand{\hastype}{{\bf~~has~type~~}}

% Definitions for \lchan and \rchan when used with an array, so that the symbol correctly
% scales to account for the dynamic space. Ideally this would require a font symbol.
\def\Lchanset{\left\{\mkern-3.5mu\left|}
\def\Rchanset{\right|\mkern-3.5mu\right\}}

\makeindex[title=Index of Semantic Rules, name=rules]
\makeindex[title=Index of Calls to Semantic Rules, name=usages]
\makeindex

\hypersetup{
	colorlinks,
	linkcolor={blue!50!black},
	citecolor={blue!50!black},
	urlcolor={blue!80!black}
}

\DeclareFontFamily{U}{mathc}{}
\DeclareFontShape{U}{mathc}{m}{it}%
{<->s*[1.03] mathc10}{}
\DeclareMathAlphabet{\mathscr}{U}{mathc}{m}{it}

%\DeclareMathAlphabet{\mathcalligra}{T1}{calligra}{m}{n}

\newcommand{\circblockbegin}{\left(\begin{array}{l}}
	\newcommand{\circblockend}{\end{array}\right)}

\newcommand{\meta}[1]{\mathsf{\color{GreyColor}\ubar{#1}}}

\makeatletter\newcommand{\linkdest}[1]{\raisebox{\ht\strutbox}{\hypertarget{#1}{}}}\makeatother
\newcommand{\linktorule}[2]{\hyperlink{#1}{#2}}

\newcounter{usage}
\newcounter{tRule}[chapter]

\newcommand{\lsem}{\lbrack\!\lbrack}
\newcommand{\rsem}{\rbrack\!\rbrack}

\usepackage{mathtools} % DeclarePairedDelimiter to define \semL 

\definecolor{ZedColor}{rgb}{0,0,0}
%\definecolor{ZedColor}{rgb}{0,0,1} % blue

\definecolor{MetaColor}{rgb}{1,0,0}
\definecolor{GreyColor}{rgb}{0.5,0.5,0.5}
\newcommand{\ubar}[1]{\mkern 1.5mu\underline{\mkern-1.5mu#1\mkern-1.5mu}\mkern 1.5mu}

%%%%%%%%%%%%%%%%%%%%%%%%%%%%%%%%%%%%%%%%%%%%%%%%%%%%%%%%%%%%%%%%%%%
%%%%%%%%%%%%%%%%%%%%%%%%%%%%%%%%%%%%%%%%%%%%%%%%%%%%%%%%%%%%%%%%%%%

%% defined rules
\newcommand{\cspdef}{\widehat=}
\newcommand{\csplet}{{\bf let~~}}
\newcommand{\cspwithin}{{\bf within~~}}

%\createrule{allSTMTransitions}
\createrule{hiddenModuleChannels}
\createrule{allTransitions}
\createrule{transitionsFrom}
\createrule[Module]{M}
\createrule{modMemory}
\createrule{constInit}
\createrule{constInitSTM}
\createrule{buildScope}
\createrule{memoryChannels}
\createrule{buffer}
\createrule{singleBuffer}
\createrule{composeControllers}
\createrule{renamingController}
\createrule{renCtrlEvts}
\createrule[Controller]{C}
\createrule{composeMachines}
\createrule{ctrlMemory}
\createrule{renamingMachine}
\createrule{renStmEvts}
\createrule{initialisation}
\createrule{composeStates}
\createrule{getsetChannels}
\createrule{trigEvents}
\createrule{stmMemory}
\createrule{getsetLocalChannels}
\createrule[State machine]{STM}
\createrule{renameTriggerEvents}

% Transitions
\createrule[Transition]{T}
\createrule[Transition]{T}[timed]

\createrule[State]{S}
\createrule{restrictedState}
\createrule{flowEvents}
\createrule{Trigger}
\createrule{triggerForMemory}
\createrule{triggerEvent}
\createrule{Action}
\createrule{flowTriggerEvents}
\createrule{exitSubstates}
\createrule{substatesTriggers}
\createrule{compileTarget}
\createrule{memoryTransition}
%
% @9/08/2021: Changed this into a normal command with no implicit cross-references, so as to enable mklatex fail when
% 			  genuine problems arise with the report.
%
%\createrule[Expression]{Expr}
\newcommand{\Expr}{Expr}

% Statements
\createrule{Statement}
\createrule{StatementInContext}
\createrule{readState}
\createrule{readState}[timed]
\createrule{usedVariables}

\createrule{allLocalVariables}
\createrule{requiredVariables}
\createrule{allVariables}
\createrule{allEvents}
\createrule{allLocalConstants}
\createrule{requiredConstants}
\createrule{requiredOperations}
\createrule{allConstants}
\createrule{states}
\createrule{reachableJunctions}
\createrule[Junction]{J}
\createrule{reachableTransitions}
\createrule{Wait}

% New rules below
\createrule{introParamaters}
\createrule{stateful}
\createrule{timedStateful}

% Memory
\createrule{sharedMemory}
\createrule{sharedConstant}
\createrule{sharedVarMemory}
\createrule{sharedConstMemory}
\createrule{varMemory}
\createrule{constMemory}
\createrule{Constant}
\createrule{Memory}

% Node Containers
\createrule{NC}
\createrule{NC}[timed]
\createrule{ncCoreBehaviour}
\createrule{ncCoreBehaviour}[timed]
\createrule{ncBehaviour}
\createrule{composeNodes}
\createrule{composeNodes}[timed]
\createrule{composeTimedNodes}
\createrule{nodeD}

% Nodes
\createrule{N}
\createrule{N}[timed]

% Transitions
\createrule{transitions}
\createrule{transitions}[timed]
\createrule{tevent}
\createrule{trigger}

% Transition deadlines
\createrule{stateDeadlines}
\createrule{transitionDeadline}

% Variable calculations
\createrule{vars}

% Read state (atomic)
\createrule{readStateA}
\createrule{readStateA}[timed]
\createrule{readStateAtomic}
\createrule{readStateAtomic}[timed]

% Clocks
\createrule{clock}[timed]
\createrule{clocks}[timed]
\createrule{clockResets}[timed]

% share-CSP
\createrule{set}
\createrule{setC}
\createrule{setL}
\createrule{setLC}
\createrule{setR}
\createrule{setRC}

\chaptercolour{blue!40!white} % Chapter heading image

%%%%%%%%%%%%%%%%%%%%%%%%%%%%%%%%%%%%%%%%%%%%%%%%%%%%%%%%%%%%%%%%%%%
% reduce vertical space before and after a verbatim environment
%%%%%%%%%%%%%%%%%%%%%%%%%%%%%%%%%%%%%%%%%%%%%%%%%%%%%%%%%%%%%%%%%%%
\usepackage{fancyvrb}
\newlength{\fancyvrbtopsep}
\newlength{\fancyvrbpartopsep}
\makeatletter
\FV@AddToHook{\FV@ListParameterHook}{\topsep=\fancyvrbtopsep\partopsep=\fancyvrbpartopsep}
\makeatother
\setlength{\fancyvrbtopsep}{0pt}
\setlength{\fancyvrbpartopsep}{0pt}

%%%%%%%%%%%%%%%%%%%%%%%%%%%%%%%%%%%%%%%%%%%%%%%%%%%%%%%%%%%%%%%%%%%
%%%%%%%%%%%%%%%%%%%%%%%%%%%%%%%%%%%%%%%%%%%%%%%%%%%%%%%%%%%%%%%%%%%

%----------------------------------------------------------------------------------------
% PREAMBLE TO PROBABILISTIC CHAPTER
%----------------------------------------------------------------------------------------

%\def \varg      {\zbinop{\zkeyword{\_}}}
\def \varg      {{{\_}}}

\newcommand{\bmapsto}{\mkern -1mu{\rightsquigarrow}\mkern -1mu} 
\newcommand{\MM}[1]{{\sffamily #1}}
\newcommand{\err}[1]{\text{Error: {#1}}}

% semL
\DeclarePairedDelimiter{\sem}{\lsem}{\rsem}
\newcommand{\semL}[2]{\sem{#1}_{\mathscr{#2}}}

\newcommand{\lsemL}{\left\lbrack\!\!\left\lbrack}
\newcommand{\rsemL}{\right\rbrack\!\!\right\rbrack}

% back to math normal mode within a meta
\newcommand{\normalinmeta}[1]{\mathnormal{\color{black}{#1}}}
\newcommand{\mathttinmeta}[1]{\mathtt{\color{black}{#1}}}
\newcommand{\prisminmath}[1]{\mathtt{\color{black}{#1}}}
\newcommand{\metanobar}[1]{\mathsf{\color{GreyColor}{#1}}}
% a new t column type will prevent italic font in an array
\newcolumntype{t}{>{\mathsf\bgroup}l<{\egroup}}
\newcommand{\metablockbegin}{\left(\begin{array}{t}}
	
	% enumitem to customise lists
	\newlist{mylist}{enumerate}{1}%
	\setlist[mylist,1]{wide = 0pt, leftmargin=*, labelindent=0pt, label=\arabic*)}
	\setlist[enumerate]{wide = 0pt, leftmargin=*, labelindent=0pt}
	\newlist{mydesc}{description}{1}%
	\setlist[mydesc,1]{wide = 0pt, leftmargin=0pt, labelindent=0pt}
	
	% for lblot and rblot
	%\DeclarePairedDelimiter{\blot}{\left\lfloor!\left\lfloor}{\right\rfloor!\right\rfloor}
	%\newcommand{\newlblot}{\left\lfloor\!\left\lfloor}
	%\newcommand{\newrblot}{\right\rfloor\!\right\rfloor}
	%\newcommand{\blot}[2]{{\newlblot{#1}\newrblot}_{\mathscr{#2}}}
	\newcommand{\newlblot}{\left\lfloor}
	\newcommand{\newrblot}{\right\rfloor}
	\newcommand{\blot}[2]{{\newlblot{#1}\newrblot}_{{#2}}}
	
%\MathSymbol{\typecolon}{\mathbin}  {symbols2}{"25}
%\newcommand{\typecolon}{\mathbin{\lhd\mkern-9mu-}} 

%\DeclareMathSymbol\typecolon{\mathbin}{AMSb}{"0223E}

%\usepackage{czt}
\def\typecolon{\raise 0.66ex\hbox{\oalign{\hfil$\scriptscriptstyle%
	\mathrm{o}$\hfil\cr\hfil$\scriptscriptstyle\mathrm{o}$\hfil}}}%
