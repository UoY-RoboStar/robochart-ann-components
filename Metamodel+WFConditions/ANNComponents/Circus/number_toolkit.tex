\documentclass[draft,a4paper,10pt,wd]{isov2}
\usepackage{ltcadiz}
\begin{document}
\normannex{Mathematical toolkit}
\sclause{Numbers}

\begin{zsection}
\SECTION number\_toolkit
\end{zsection}

\ssclause{Successor}

\begin{zed}
\function (succ~\_)
\end{zed}

\begin{axdef}
succ~\_ : \power (\nat \cross \nat)
\where
(succ~\_) = \lambda n : \nat @ n+1
\end{axdef}

The successor of a natural number $n$ is equal to $n+1$.

\ssclause{Integers}

%%Zchar \num U+2124
\begin{axdef}
\num : \power \arithmos
\end{axdef}

$\num$ is the set of integers,
that is, positive and negative whole numbers and zero.
The set $\num$ is characterised by axioms for its additive structure given
in the prelude (clause \ref{sec:prelude}) together with
the next formal paragraph below.

Number systems that extend the integers
may be specified as supersets of $\num$.

\ssclause{Addition of integers, arithmetic negation}

%%Zprechar \negate U+002D
\begin{zed}
\function (\negate \_)
\end{zed}

\begin{axdef}
\negate \_ : \power (\arithmos \cross \arithmos)
\where
\forall x, y : \num @ \exists_1 z : \num @ ((x,y),z) \in (\_ + \_)\\
\forall x : \num @ \exists_1 y : \num @ (x,y) \in (\negate \_)
\also
\forall i , j , k : \num @
\\ \t1	( i + j ) + k = i + ( j + k )
\\ \t1	\land i + j = j + i
\\ \t1	\land i + \negate i = 0
\\ \t1	\land i + 0 = i
\also
\num = \{ z : \arithmos | \exists x : \nat @ z = x \lor z = \negate x \}
\end{axdef}

The binary addition operator $(\_ + \_)$ is defined in the prelude
(clause~\ref{sec:prelude}).
The definition here introduces additional properties for integers.
The addition and negation operations on integers are total functions that
take integer values.
The integers form a commutative group under $(\_ + \_)$ with $\negate$ as
the inverse operation and $0$ as the identity element.
%%quiet 1
%%reckless 1
\begin{note}
If $function\_toolkit$ notation were exploited,
the negation operator could be defined as follows.
%\begin{axdef}
%\negate \_ : \arithmos \pfun \arithmos
%\where
%( \num \cross \num ) \dres ( \_ + \_ ) \in \num \cross \num \fun \num
%\also
%\num \dres \negate \in \num \fun \num
%\also
%\forall i , j , k : \num @
%\\ \t1  ( i + j ) + k = i + ( j + k )
%\\ \t1  \land i + j = j + i
%\\ \t1  \land i + \negate i = 0
%\\ \t1  \land i + 0 = i
%\also
%\forall h : \power \num @
%\\ \t1  1 \in h
%        \land ( \forall i , j : h @ i + j \in h \land \negate i \in h )
%\\ \t2          \implies h = \num
%\end{axdef}
\end{note}
%%reckless 0
%%quiet 0

\ssclause{Subtraction}

\begin{zed}
\function 30 \leftassoc (\_ - \_)
\end{zed}

\begin{axdef}
\_ - \_ : \power ((\arithmos \cross \arithmos) \cross \arithmos)
\where
\forall x, y : \num @ \exists_1 z : \num @ ((x,y),z) \in (\_ - \_)
\also
\forall i , j : \num @ i - j = i + \negate j
\end{axdef}

Subtraction is a function whose domain includes all pairs of integers.
For all integers $i$ and $j$,
$i - j$ is equal to $i + \negate j$.
%%quiet 1
%%reckless 1
\begin{note}
If $function\_toolkit$ notation were exploited,
the subtraction operator could be defined as follows.
%\begin{axdef}
%\_ - \_ : \arithmos \cross \arithmos \pfun \arithmos
%\where
%( \num \cross \num ) \dres ( \_ - \_ ) \in \num \cross \num \fun \num
%\also
%\forall i , j : \num @ i - j = i + \negate j
%\end{axdef}
\end{note}
%%reckless 0
%%quiet 0

\ssclause{Less-than-or-equal}

%%Zinchar \leq U+2264
\begin{zed}
\relation (\_ \leq \_)
\end{zed}

\begin{axdef}
\_ \leq \_ : \power (\arithmos \cross \arithmos)
\where
\forall i , j : \num @ i \leq j \iff j - i \in \nat
\end{axdef}

For all integers $i$ and $j$,  $i \leq j$ if and only if
their difference $j - i$ is a natural number.

\ssclause{Less-than}

\begin{zed}
\relation (\_ < \_)
\end{zed}

\begin{axdef}
\_ < \_ : \power (\arithmos \cross \arithmos)
\where
\forall i , j : \num @ i < j \iff i + 1 \leq j
\end{axdef}

For all integers $i$ and $j$,
$i < j$ if and only if $i + 1 \leq j$.

\ssclause{Greater-than-or-equal}

%%Zinchar \geq U+2265
\begin{zed}
\relation (\_ \geq \_)
\end{zed}

\begin{axdef}
\_ \geq \_ : \power (\arithmos \cross \arithmos)
\where
\forall i , j : \num @ i \geq j \iff j \leq i
\end{axdef}

For all integers $i$ and $j$,
$i \geq j$ if and only if $j \leq i$.

\ssclause{Greater-than}

\begin{zed}
\relation (\_ > \_)
\end{zed}

\begin{axdef}
\_ > \_ : \power (\arithmos \cross \arithmos)
\where
\forall i , j : \num @ i > j \iff j < i
\end{axdef}

For all integers $i$ and $j$,
$i > j$ if and only if $j < i$.

\ssclause{Strictly positive natural numbers}

\begin{zed}
\nat_1 == \{ x : \nat | \lnot x = 0 \}
\end{zed}

The strictly positive natural numbers $\nat_1$
are the natural numbers except zero. 

\ssclause{Non-zero integers}

\begin{zed}
\num_1 == \{ x : \num | \lnot x = 0 \}
\end{zed}

The non-zero integers $\num_1$ are the integers except zero. 

\ssclause{Multiplication of integers}

\begin{zed}
\function 40 \leftassoc (\_ * \_)
\end{zed}

\begin{axdef}
\_ * \_ : \power ((\arithmos \cross \arithmos) \cross \arithmos)
\where
\forall x, y : \num @ \exists_1 z : \num @ ((x,y),z) \in (\_ * \_)
\also
\forall i , j , k : \num @
\\ \t1	( i * j ) * k = i * ( j * k )
\\ \t1	\land i * j = j * i
\\ \t1	\land i * ( j + k ) = i * j + i * k
\\ \t1	\land 0 * i = 0
\\ \t1	\land 1 * i = i
\end{axdef}

The binary multiplication operator $(\_ * \_)$ is defined for integers.
The multiplication operation on integers
is a total function and has integer values.
Multiplication on integers is characterised by the unique operation under
which the integers become a commutative ring with identity element $1$.
%%quiet 1
%%reckless 1
\begin{note}
If $function\_toolkit$ notation were exploited,
the multiplication operator could be defined as follows.
%\begin{axdef}
%\_ * \_ : (\arithmos \cross \arithmos) \pfun \arithmos
%\where
%( \num \cross \num ) \dres ( \_ * \_ ) \in \num \cross \num \fun \num
%\also
%\forall i , j , k : \num @
%\\ \t1  ( i * j ) * k = i * ( j * k )
%\\ \t1  \land i * j = j * i
%\\ \t1  \land i * ( j + k ) = i * j + i * k
%\\ \t1  \land 0 * i = 0
%\\ \t1  \land 1 * i = i
%\end{axdef}
\end{note}
%%reckless 0
%%quiet 0

\ssclause{Division, modulus}

%%Zinword \div div
%%Zinword \mod mod
\begin{zed}
\function 40 \leftassoc (\_ \div \_)\\
\function 40 \leftassoc (\_ \mod \_)
\end{zed}

\begin{axdef}
\_ \div \_~, \_ \mod \_ : \power ((\arithmos \cross \arithmos) \cross \arithmos)
\where
\forall x : \num ; y : \num_1 @ \exists_1 z : \num @ ((x,y),z) \in (\_ \div \_)
\also
\forall x : \num ; y : \num_1 @ \exists_1 z : \num @ ((x,y),z) \in (\_ \mod \_)
\also
\forall i : \num ; j : \num_1 @
\\ \t1	i = ( i \div j ) * j + i \mod j
\\ \t1	\land ( 0 \leq i \mod j < j  \lor  j < i \mod j \leq 0 )
\end{axdef}

For all integers $i$ and non-zero integers $j$,
the pair $(i, j)$ is in the domain of $\_ div \_$ and of $\_ mod \_$,
and $i \div j$ and $i \mod j$ have integer values.

When not zero, $i \mod j$ has the same sign as $j$.
This means that $i \div j$ is the largest integer no greater
than the rational number $i/j$.
%%quiet 1
%%reckless 1
\begin{note}
If $function\_toolkit$ notation were exploited,
the division and modulus operators could be defined as follows.
%\begin{axdef}
%\_ \div \_~, \_ \mod \_ : \arithmos \cross \arithmos \pfun \arithmos
%\where
%( \num \cross \num_1 ) \dres ( \_ \div \_ ) \in \num \cross \num_1 \fun \num
%\also
%( \num \cross \num_1 ) \dres ( \_ \mod \_ ) \in \num \cross \num_1 \fun \num
%\also
%\forall i : \num ; j : \num_1 @
%\\ \t1  i = ( i \div j ) * j + i \mod j
%\\ \t1  \land ( 0 \leq i \mod j < j  \lor  j < i \mod j \leq 0 )
%\end{axdef}
\end{note}
%%reckless 0
%%quiet 0

\end{document}
